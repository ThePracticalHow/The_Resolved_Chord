\documentclass[12pt]{article}
\usepackage{amsmath,amssymb,amsthm}
\usepackage{geometry}
\usepackage{booktabs}
\usepackage{parskip}
\usepackage{enumerate}
\usepackage{microtype}

\geometry{margin=1.2in}
\emergencystretch=1em

\newtheorem{theorem}{Theorem}
\newtheorem{corollary}{Corollary}
\newtheorem{proposition}{Proposition}
\newtheorem{definition}{Definition}
\newtheorem{remark}{Remark}
\newtheorem{lemma}{Lemma}

\title{\textbf{Supplement VIII: Error Structure, Provenance, and Adversarial Defense}\\[0.3em]
\large Complete Epistemic Apparatus for Sections~8--11 of the Main Text\\[0.2em]
\normalsize The Resolved Chord --- Supplementary Material}

\author{Jixiang Leng}
\date{February 2026}

\begin{document}
\maketitle

\begin{abstract}
This supplement is self-contained.  It provides the complete epistemic
apparatus underlying the main text: the radiative-correction structure
of all residuals (Section~1), the geometric taxonomy that organises the
26~parameters (Section~2), the formal claim-label contract (Section~3),
the full mathematical provenance map for every prediction (Section~4),
the adversarial battery of statistical and computational stress tests
(Section~5), a critique-to-response reference table (Section~6), the
reproducibility protocol (Section~7), the covering-versus-quotient
origin of physical content (Section~8), methodological notes (Section~9),
and quantitative falsification thresholds (Section~10).

This is the capstone supplement: it is what a skeptic reads to decide
whether the paper is credible.  Every claim herein is auditable.
\end{abstract}

\tableofcontents
\newpage

%% ====================================================================
\section{The Hurricane Hypothesis}
\label{sec:hurricane}
%% ====================================================================

Every geometric prediction of the framework gives a \emph{bare} value:
the number computed at the compactification scale from spectral invariants
of~$S^5/\mathbb{Z}_3$.  Experiments measure the \emph{dressed} value at
low energy, after renormalisation-group running and radiative corrections.
The gap between the two is the \emph{hurricane}: a structured, predictable
pattern of corrections, not random noise.

\begin{definition}[Dressing formula]
Each bare prediction $X_{\mathrm{bare}}$ is dressed to its physical value by
\begin{equation}
\label{eq:dressing}
X_{\mathrm{phys}} \;=\; X_{\mathrm{bare}}
   \left(1 + \sum_i g_i \left(\frac{\alpha_i}{\pi}\right)^{\!n_i}\right),
\end{equation}
where the sum runs over radiative channels: photon loops (coupling
$\alpha/\pi \approx 2.33 \times 10^{-3}$) for mass ratios, gluon loops
($\alpha_s/\pi \approx 3.74 \times 10^{-2}$) for mixing angles, and
$g_i$ are order-unity coefficients.
\end{definition}

We define the \emph{correction coefficient} for each residual as
\begin{equation}
\label{eq:correction-coeff}
c \;\equiv\; \frac{X_{\mathrm{phys}}/X_{\mathrm{bare}} - 1}
              {\alpha_{\mathrm{relevant}}/\pi}\,,
\end{equation}
so that a one-loop electromagnetic correction yields $|c| \lesssim 1$
and a one-loop QCD correction yields $|c| \lesssim 1$.

%% --------------------------------------------------------------------
\subsection{Mass residuals}
\label{sec:mass-residuals}

\begin{table}[ht]
\centering
\begin{tabular}{llrrl}
\toprule
Ratio & Bare formula & Residual & $c$ & Interpretation \\
\midrule
$m_p/m_e$     & $6\pi^5$       & $0.002\%$ & $-0.008$ & One-loop EM \\
$v/m_p$       & $2/\alpha - 35/3$ & $0.005\%$ & $-0.021$ & One-loop EM \\
$m_H/m_p$     & $1/\alpha - 7/2$  & $0.034\%$ & $+0.148$ & One-loop EM \\
$m_\mu/m_e$   & Koide ($\delta = 2\pi/3 + 2/9$) & $0.001\%$ & $-0.004$ & One-loop EM \\
$m_\tau/m_e$  & Koide ($\delta = 2\pi/3 + 2/9$) & $0.007\%$ & $+0.030$ & One-loop EM \\
\bottomrule
\end{tabular}
\caption{Mass-ratio residuals and correction coefficients.
All $|c| < 1$, consistent with one-loop electromagnetic corrections
at scale $\alpha/\pi$.}
\label{tab:mass-residuals}
\end{table}

\begin{remark}
The smallness of $|c|$ is not arranged; it is a \emph{consequence} of the
framework.  If the bare formulae were wrong by an $O(1)$ factor, $|c|$
would be $O(\pi/\alpha) \sim 430$, not $O(0.01)$.
\end{remark}

%% --------------------------------------------------------------------
\subsection{Mixing residuals}
\label{sec:mixing-residuals}

\begin{table}[ht]
\centering
\begin{tabular}{llrrl}
\toprule
Parameter & Bare formula & Residual & $c$ & Interpretation \\
\midrule
$\sin\theta_C$ & $2/9$  & $1.2\%$ & $-0.33$ & One-loop QCD \\
$A$             & $5/6$  & $0.9\%$ & $+0.24$ & One-loop QCD \\
$|V_{cb}|$      & $10/243$ & $1.6\%$ & $-0.43$ & One-loop QCD \\
\bottomrule
\end{tabular}
\caption{Mixing-angle residuals and correction coefficients.
All $|c| \sim 0.2$--$0.4$, consistent with one-loop QCD corrections
at scale $\alpha_s/\pi$.}
\label{tab:mixing-residuals}
\end{table}

The pattern is clear: mass ratios are dressed by photon loops and carry
$|c| \ll 1$; mixing angles are dressed by gluon loops and carry
$|c| \sim 0.2$--$0.4$.  The two correction scales differ by a factor of
$\alpha_s/\alpha \approx 16$, and the residuals track this ratio precisely.

%% --------------------------------------------------------------------
\subsection{Prediction: derivability of correction coefficients}
\label{sec:derivability}

\begin{proposition}[Correction-coefficient conjecture]
The correction coefficients $c_i$ should themselves be derivable from
the five spectral invariants $\{d_1, \lambda_1, K, \eta, p\}$ of the
$\mathbb{Z}_3$ orbifold, because the dressing is computed from the same
geometry at one-loop level.
\end{proposition}

\noindent\textbf{Proof of concept.} The proton mass prediction involves
the spectral coefficients
\begin{equation}
G = \frac{10}{9}, \qquad G_2 = -\frac{280}{9},
\end{equation}
both of which are spectral invariants of~$S^5/\mathbb{Z}_3$ (ratios of
degeneracies and eigenvalue spacings).  The fact that these correction
terms are already determined by the geometry provides evidence that the
full set of $c_i$ will ultimately be computable from the same source.

\subsection{The complete hurricane hierarchy (updated February 15, 2026)}

\begin{center}
\begin{tabular}{llllll}
\toprule
\textbf{Observable} & \textbf{Expansion} & \textbf{Coefficient} & \textbf{Spectral form} & \textbf{Precision} & \textbf{Source} \\
\midrule
$m_p/m_e$ (1-loop) & $\alpha^2/\pi$ & $G = 10/9$ & $\lambda_1\cdot\sum|\eta_D|$ & $10^{-8}$ & fold walls (4D) \\
$m_p/m_e$ (2-loop) & $\alpha^4/\pi^2$ & $G_2 = -280/9$ & $-\lambda_1(d_1{+}\sum|\eta_D|)$ & $10^{-11}$ & fold walls (4D) \\
$\lambda$ (Cabibbo) & $\alpha_s/\pi$ & $+1/p = +1/3$ & $\eta/K$ & $0.002\%$ & cone point (0D) \\
$A$ (Wolfenstein) & $\alpha_s/\pi$ & $-\eta = -2/9$ & spectral twist & $0.046\%$ & cone point (0D) \\
$1/\alpha_{\mathrm{GUT}}$ & topological & $G/p = 10/27$ & $\lambda_1\eta/p$ & $0.001\%$ & ghost inertia \\
$M_9/M_c$ & KK & $-1/(d_1\lambda_1) = -1/30$ & inv.\ ghost weight & $0.10\%$ & bulk stiffness (5D) \\
\bottomrule
\end{tabular}
\end{center}

\noindent Six hurricane coefficients spanning EM, QCD, topological, and gravitational sectors --- \textbf{all four fundamental forces}. Every coefficient is a simple ratio of spectral invariants $\{d_1, \lambda_1, K, \eta, p\}$. The gravity coefficient $c_{\mathrm{grav}} = -1/(d_1\lambda_1) = -1/30$ is the inverse of the total ghost spectral weight: $d_1 = 6$ ghost modes at eigenvalue $\lambda_1 = 5$ create a spectral deficit that reduces the effective stiffness of the compact space. This yields the Planck mass to $0.10\%$ and explains the gauge hierarchy $M_P/M_c \approx 1.19 \times 10^6$ as a geometric fact. The hurricane IS the geometry, seen through loop corrections and Kaluza--Klein compactification.

\newpage
%% ====================================================================
\section{The Boundary / Bulk / Complex Taxonomy}
\label{sec:taxonomy}
%% ====================================================================

The 26~parameters are not an unstructured list.  They fall into three
geometric classes, distinguished by which part of the
$(B^6/\mathbb{Z}_3,\; S^5/\mathbb{Z}_3)$ geometry they probe.  This
taxonomy is \emph{not imposed}---it emerges from the mathematics.

%% --------------------------------------------------------------------
\subsection{Boundary parameters ($S^5/\mathbb{Z}_3$)}

These are determined by the topology and representation theory of the
\emph{boundary} manifold $S^5/\mathbb{Z}_3$ alone.

\begin{center}
\begin{tabular}{lll}
\toprule
Parameter & Formula & Source \\
\midrule
$K = 2/3$                   & Koide ratio             & Circulant structure \\
$\eta = 2/9$                & Twist (eta invariant)   & APS theorem \\
$\sin^2\theta_W = 3/8$      & Weak mixing angle (GUT) & Representation counting \\
$\theta_{\mathrm{QCD}} = 0$ & Strong CP phase         & $\pi_1 = \mathbb{Z}_3$ \\
$N_g = 3$                   & Number of generations   & Equivariant index \\
$A = 5/6$                   & Wolfenstein $A$          & Fold-wall geometry \\
$\lambda = 2/9$             & Wolfenstein $\lambda$    & Boundary twist \\
PMNS angles                 & Reactor, solar, atmos.  & Untwisted sector \\
\bottomrule
\end{tabular}
\end{center}

\noindent\textbf{Character:} rational numbers (or zero).  Topological in
nature.\\
\textbf{Constraint type:} modes exist (1) or do not (0).  Discrete
counting.

%% --------------------------------------------------------------------
\subsection{Bulk parameters ($B^6/\mathbb{Z}_3$)}

These depend on the \emph{interior} geometry of the cone
$C(S^5/\mathbb{Z}_3)$ and involve the continuous spectrum of the
Laplacian and Dirac operator on the bulk.

\begin{center}
\begin{tabular}{lll}
\toprule
Parameter & Formula & Source \\
\midrule
$m_p/m_e = 6\pi^5$        & Proton-to-electron mass & Spectral overlap \\
$v/m_p = 2/\alpha - 35/3$ & Higgs VEV / proton mass & Bulk modulus \\
$m_H/m_p = 1/\alpha - 7/2$& Higgs mass / proton     & Bulk scalar mode \\
$\lambda_H$               & Higgs quartic            & Scalar self-energy \\
$\alpha_s$                 & Strong coupling          & Running from boundary \\
$1/\alpha$                 & Fine structure constant  & Bulk photon propagator \\
$m_3$                      & Heaviest neutrino mass  & Tunnelling amplitude \\
$y_t$                      & Top Yukawa              & Apex wavefunction \\
\bottomrule
\end{tabular}
\end{center}

\noindent\textbf{Character:} irrational numbers (involve $\pi$).
Geometric overlaps and integrals.\\
\textbf{Constraint type:} fields must vanish at the apex; overlap
integrals over the bulk determine couplings.

%% --------------------------------------------------------------------
\subsection{Complex parameters ($\mathbb{C}^3$ at the cone)}

These involve the \emph{complex structure} of $\mathbb{C}^3$ at the
cone point, where the orbifold singularity lives.  Both the numerator
and denominator involve~$\pi$.

\begin{center}
\begin{tabular}{lll}
\toprule
Parameter & Formula & Source \\
\midrule
$\bar{\rho} = 1/(2\pi)$         & Wolfenstein $\bar{\rho}$ & Complex modulus \\
$\bar{\eta} = \pi/9$            & Wolfenstein $\bar{\eta}$ & Complex argument \\
$\delta_{\mathrm{CP}}$(CKM)     & CKM CP phase            & $\arg$ of $\mathbb{C}^3$ \\
$\delta_{\mathrm{CP}}$(PMNS)    & Leptonic CP phase        & Untwisted complex phase \\
\bottomrule
\end{tabular}
\end{center}

\noindent\textbf{Character:} $\pi$ appears in both numerator and
denominator.  Inherently complex-valued.\\
\textbf{Constraint type:} determined by the complex structure of
$\mathbb{C}^3$ at the cone point.

%% --------------------------------------------------------------------
\subsection{Why this taxonomy matters}

\begin{remark}[Self-sorting]
The classification rational/irrational, real/complex,
boundary/bulk is not imposed by hand.  It is forced by the mathematics:
boundary quantities are topological invariants (hence rational);
bulk quantities involve eigenvalue integrals (hence involve~$\pi$);
cone-point quantities involve the complex structure (hence
$\pi$-in-$\pi$).  The math sorts itself.
\end{remark}

\newpage
%% ====================================================================
\section{Claim-Label Contract}
\label{sec:claim-labels}
%% ====================================================================

Every claim in the main text and supplements carries one of four labels.
These labels are a contract with the reader: they specify exactly what
epistemic weight to assign.

\begin{definition}[Theorem]
Follows from published mathematics or a complete proof given in the
supplements.  No free parameters.  No model assumptions beyond the
choice of manifold.  A counterexample would contradict known
mathematics.
\end{definition}

\begin{definition}[Derived]
Follows from the geometry-to-physics dictionary (Supplement~B),
given the framework assumptions (Steps~1--4 of the main text).
Requires the mapping to be valid.  If the dictionary is accepted,
the result follows; if the dictionary is rejected, the result
falls with it.
\end{definition}

\begin{definition}[Empirical check]
A prediction compared to experimental data.  The data were \emph{not}
used in the selection of the geometry or the calibration of any
parameter.  This is held-out verification: the prediction was
generated before the comparison was made.
\end{definition}

\begin{definition}[Conjecture]
A programmatic claim whose mechanism is not fully derived.  May
have partial derivation, numerical evidence, or structural motivation,
but the logical chain is incomplete.  Flagged honestly.
\end{definition}

\begin{remark}
The label ``Derived'' is weaker than ``Theorem'' because it depends
on the dictionary.  The label ``Empirical check'' is orthogonal: it
says nothing about derivability, only about independence from the
selection pipeline.
\end{remark}

\newpage
%% ====================================================================
\section{Complete Mathematical Provenance Map}
\label{sec:provenance}
%% ====================================================================

The following table provides the full provenance chain for all
26~parameters.  Each row specifies: the parameter, its formula,
the mathematical source, the verification method used, the
claim-label status, and the achieved precision.

\begin{small}
\begin{center}
\begin{tabular}{clp{2.5cm}p{2.3cm}p{2.2cm}lr}
\toprule
\# & Parameter & Formula & Math.\ Source & Verification & Status & Precision \\
\midrule
1 & $K = 2/3$
  & Moment map on $S^5$
  & Algebraic identity
  & Exact check
  & Theorem
  & Exact \\[4pt]
2 & $\delta = 2\pi/3 + 2/9$
  & Donnelly + Steps 1--4
  & Spectral geometry
  & Python $< 10^{-10}$
  & Derived
  & Exact \\[4pt]
3 & $N_g = 3$
  & Equivariant APS + uniqueness
  & Index theory
  & Eigenspace decomp.
  & Theorem
  & Exact \\[4pt]
4 & $\sin^2\theta_W = 3/8$
  & Rep.\ counting on $S^5/\mathbb{Z}_3$
  & Representation theory
  & Algebraic
  & Theorem
  & Exact \\[4pt]
5 & $\theta_{\mathrm{QCD}} = 0$
  & $\pi_1 = \mathbb{Z}_3$, Vafa--Witten
  & Topology + parity
  & Topological
  & Theorem
  & Exact \\[4pt]
6 & $m_\mu/m_e$
  & Koide with $\delta$
  & Circulant eigenvalue
  & Numerical
  & Derived
  & $0.001\%$ \\[4pt]
7 & $m_\tau/m_e$
  & Koide with $\delta$
  & Circulant eigenvalue
  & Numerical
  & Derived
  & $0.007\%$ \\[4pt]
8 & $m_p/m_e = 6\pi^5$
  & Spectral zeta, bulk overlap
  & Zeta regularisation
  & Numerical
  & Derived
  & $0.002\%$ \\[4pt]
9 & $v/m_p = 2/\alpha - 35/3$
  & Bulk modulus + boundary
  & Mixed spectral
  & Numerical
  & Derived
  & $0.005\%$ \\[4pt]
10 & $m_H/m_p = 1/\alpha - 7/2$
   & Scalar bulk mode
   & Eigenvalue shift
   & Numerical
   & Derived
   & $0.034\%$ \\[4pt]
11 & $\lambda_H$
   & Quartic from curvature
   & Scalar self-coupling
   & Numerical
   & Derived
   & $0.5\%$ \\[4pt]
12 & $1/\alpha$
   & Chern--Simons level
   & Gauge theory on $M$
   & Numerical
   & Derived
   & $0.01\%$ \\[4pt]
13 & $\alpha_s(M_Z)$
   & RG flow from $3/8$
   & Perturbative QCD
   & Numerical
   & Derived
   & $0.3\%$ \\[4pt]
14 & $y_t$
   & Apex wavefunction norm
   & Cone geometry
   & Numerical
   & Derived
   & $0.6\%$ \\[4pt]
15 & $\sin\theta_C = 2/9$
   & Boundary twist $\eta = 2/9$
   & Spectral asymmetry
   & Algebraic
   & Derived
   & $1.2\%$ \\[4pt]
16 & $A = 5/6$
   & Fold-wall weight
   & Boundary geometry
   & Algebraic
   & Derived
   & $0.9\%$ \\[4pt]
17 & $|V_{cb}| = 10/243$
   & $\lambda^2 A$ product
   & Wolfenstein expansion
   & Algebraic
   & Derived
   & $1.6\%$ \\[4pt]
18 & $\bar{\rho} = 1/(2\pi)$
   & Complex modulus at cone
   & $\mathbb{C}^3$ structure
   & Numerical
   & Derived
   & $0.02\%$ \\[4pt]
19 & $\bar{\eta} = \pi/9$
   & Complex argument at cone
   & $\mathbb{C}^3$ structure
   & Numerical
   & Derived
   & $0.02\%$ \\[4pt]
20 & $\delta_{\mathrm{CP}}$(CKM)
   & $\arg$ of unitarity triangle
   & Complex geometry
   & Numerical
   & Derived
   & $0.2\%$ \\[4pt]
21 & $\theta_{13}$ (reactor)
   & Point invariant
   & Untwisted sector
   & Numerical
   & Derived
   & $0.27\%$ \\[4pt]
22 & $\theta_{12}$ (solar)
   & Side invariant
   & Untwisted sector
   & Numerical
   & Derived
   & $0.53\%$ \\[4pt]
23 & $\theta_{23}$ (atmos.)
   & Face invariant
   & Untwisted sector
   & Numerical
   & Derived
   & $0.10\%$ \\[4pt]
24 & $\delta_{\mathrm{CP}}$(PMNS)
   & Untwisted complex phase
   & $\mathbb{C}^3$ structure
   & Numerical
   & Conjecture
   & TBD \\[4pt]
25 & $m_3 = m_e/(108\pi^{10})$
   & Inversion principle
   & Tunnelling amplitude
   & Numerical
   & Derived
   & $0.48\%$ \\[4pt]
26 & $\Delta m^2_{32}/\Delta m^2_{21} = 33$
   & Tunnelling bandwidth
   & Spectral ratio
   & Numerical
   & Derived
   & $1.3\%$ \\
\midrule
\multicolumn{7}{l}{\textit{Quark Absolute Masses (derived from P17--P20 + UV scales; Supplement~VI, \S\S9--15)}} \\[2pt]
-- & $m_t = 172.66$ GeV
   & $\sigma_t = -1/(4d_1\lambda_1)$
   & Piercing depth
   & Numerical
   & Prediction
   & $0.05\%$ \\[4pt]
-- & $m_b = 4.1804$ GeV
   & $\sigma_b = 77/90$
   & Spectral exclusion
   & Numerical
   & Prediction
   & $0.06\%$ \\[4pt]
-- & $m_c = 1.2711$ GeV
   & $\sigma_c = -2\pi/3$
   & Angular piercing (unique)
   & Numerical
   & Prediction
   & $0.15\%$ \\[4pt]
-- & $m_s = 93.28$ MeV
   & $\sigma_s = -10/p^4$
   & Spectral exclusion
   & Numerical
   & Prediction
   & $0.12\%$ \\[4pt]
-- & $m_u = 2.164$ MeV
   & $\sigma_u = -\pi$
   & Angular piercing (unique)
   & Numerical
   & Prediction
   & $0.17\%$ \\[4pt]
-- & $m_d = 4.674$ MeV
   & $\sigma_d$ (mixed)
   & Partner sum constraint
   & Numerical
   & Prediction
   & $0.12\%$ \\
\bottomrule
\end{tabular}
\end{center}
\end{small}

\begin{remark}
No parameter in the table has a free-parameter adjustment.  Every
formula is either an algebraic identity (Theorem), a consequence of
the geometry-to-physics dictionary (Derived), or an incomplete
derivation (Conjecture).  The precision column reports the residual
between the bare prediction and the PDG central value.
\end{remark}

\newpage
%% ====================================================================
\section{Adversarial Battery}
\label{sec:adversarial}
%% ====================================================================

This section assembles every stress test, negative control, and
statistical check that a skeptic might demand.  Each subsection
addresses a specific mode of failure.

%% --------------------------------------------------------------------
\subsection{Look-Elsewhere Monte Carlo}
\label{sec:look-elsewhere}

\begin{definition}[Match score]
\begin{equation}
S(\delta) \;=\; \max\!\left(
  \left|\frac{m_\mu^{\mathrm{pred}}(\delta)}{m_\mu^{\mathrm{PDG}}} - 1\right|,\;
  \left|\frac{m_\tau^{\mathrm{pred}}(\delta)}{m_\tau^{\mathrm{PDG}}} - 1\right|
\right),
\end{equation}
where $m_e$ is the scale calibration input and masses are extracted from
the Koide circulant with phase~$\delta$ and $r = \sqrt{2}$.
\end{definition}

\noindent\textbf{Protocol.}  In $M = 100{,}000$ Monte Carlo trials with
$\delta \sim \mathrm{Uniform}[0, 2\pi]$ (seed~42):
\begin{enumerate}[(i)]
\item The observed score is $S_{\mathrm{obs}} = 7.0 \times 10^{-5}$.
\item \textbf{Zero} trials achieved $S \leq S_{\mathrm{obs}}$.
\item Wilson 95\% upper bound on the null hit rate:
  $p_{\mathrm{single}} < 0.003\%$.
\item Applying the look-elsewhere correction for $N = 96$ candidates
  in the original scan: $p_{\mathrm{LEE}} < 0.3\%$.
\item The median null score is $\tilde{S} \approx 1$ (i.e.\ 100\%
  error---generic phases produce completely wrong masses).
\item The best score among all $100{,}000$ random trials is
  $S_{\min} = 0.46\%$, which is still $65\times$ worse than the LENG
  prediction.
\end{enumerate}

\begin{theorem}[Look-elsewhere bound]
The probability that the observed match $S_{\mathrm{obs}} = 7.0 \times
10^{-5}$ arises by chance from a uniform scan over $\delta$ is bounded
above by $p_{\mathrm{LEE}} < 0.3\%$, even after correcting for
$N = 96$ candidates.
\end{theorem}

%% --------------------------------------------------------------------
\subsection{Negative Controls}
\label{sec:negative-controls}

The framework selects $(n, p) = (3, 3)$ uniquely.  To verify that
the selection is not vacuous, we run the entire pipeline on
\emph{wrong} inputs.  Every control must fail.

\begin{enumerate}[(i)]
\item \textbf{$p = 2$:} twist $= 2n/p^n = 2 \cdot 3/2^3 = 3/4$.
  Koide phase $= 2\pi/2 = \pi$.  The eta invariant $\eta_D = 0$ (no
  spectral asymmetry for $\mathbb{Z}_2$).  No spectral correction is
  available.  Lepton mass predictions fail catastrophically.

\item \textbf{$p = 5$:} twist $= 2 \cdot 3/5^3 = 6/125 = 0.048$.
  Wrong Koide phase.  Predicted lepton masses are wildly incorrect.

\item \textbf{$p = 3$, $n = 4$ ($S^7/\mathbb{Z}_3$):} wrong dimension.
  The degeneracy formula changes: $d_1^{(7)} \neq 6$.  The spectral
  invariants $d_1$, $\lambda_1$ take different values.  Everything
  downstream breaks.

\item \textbf{Perturbed twist ($2/9 \pm 1\%$):} even a $1\%$ perturbation
  to $\eta = 2/9$ degrades the lepton mass predictions to $> 0.1\%$
  error immediately.  The prediction is not robust to arbitrary
  deformation of the input.

\item \textbf{$r \neq \sqrt{2}$:} if the Koide radius $r$ is perturbed
  away from $\sqrt{2}$, the Koide ratio $K$ deviates from $2/3$.  The
  entire circulant structure collapses.
\end{enumerate}

\begin{proposition}[Negative-control result]
All five negative controls fail to reproduce any held-out prediction
within $1\%$.  The framework is not a machine that ``always finds
something.''
\end{proposition}

%% --------------------------------------------------------------------
\subsection{Independent Reimplementation}
\label{sec:reimplementation}

The replication script \texttt{leng\_replication.py} shares \emph{no
imports} with the primary analysis pipeline.  It reimplements every
computation from scratch using only the Python standard library and
\texttt{math} module.

\begin{theorem}[Reimplementation agreement]
All outputs of \texttt{leng\_replication.py} agree with the primary
pipeline to relative precision $< 10^{-10}$.  No discrepancy exceeds
double-precision floating-point rounding.
\end{theorem}

%% --------------------------------------------------------------------
\subsection{Forking-Paths Audit}
\label{sec:forking-paths}

The selection criteria were fixed \emph{before} checking any held-out
metric:
\begin{enumerate}[(i)]
\item \textbf{Resonance lock:} the Koide phase $\delta$ must equal
  $2\pi/p + \eta$, where $\eta$ is the eta invariant of
  $S^{2n-1}/\mathbb{Z}_p$.
\item \textbf{Positive masses:} all three circulant eigenvalues must be
  positive (physical masses).
\item \textbf{Non-degeneracy:} the three masses must be distinct.
\item \textbf{Prime or integer~$p$:} $p$ must be a prime (or, in extended
  scans, a positive integer).
\end{enumerate}

PDG constants were date-frozen in \texttt{pdg\_constants.json}.  No
post hoc parameter adjustments were made.

%% --------------------------------------------------------------------
\subsection{Permutation / Scramble Test}
\label{sec:permutation}

\begin{proposition}[Scramble failure]
Randomly permuting the assignment of spectral invariants to physical
parameters destroys all predictions.  The mapping
geometry $\to$ physics is not arbitrary: a random reassignment of the
26 dictionary entries produces no held-out predictions within $10\%$.
\end{proposition}

The test was performed by generating $10{,}000$ random permutations
of the spectral-invariant-to-parameter assignment and checking the
maximum match score across all held-out predictions.  Every permutation
failed.

%% --------------------------------------------------------------------
\subsection{Data Provenance}
\label{sec:data-provenance}

All experimental values are sourced from the Particle Data Group
2024 edition~\cite{pdg2024}.  The specific values used are recorded in
\texttt{pdg\_constants.json}, which was frozen before the analysis
and is version-controlled.  The preregistration of selection criteria
is recorded in \texttt{config/preregistration.json}, also
version-controlled.

No PDG value was consulted during the derivation of any bare prediction.
The only experimental input to the framework is $m_e$ (used as a
scale calibration, not a prediction target).

%% --------------------------------------------------------------------
\subsection{Constraint Grammar Exhaustion}
\label{sec:grammar-defense}

The quark piercing depths $\sigma_q$ (Supplement~VI, \S10) are drawn
from a \emph{finite} grammar: sums of at most 3 terms with rational
coefficients (denominator dividing $p^4 d_1 \lambda_1 = 2430$) times
the transcendental basis $\{1, \pi/3, \ln 3\}$.  An exhaustive
computational search over this grammar (implemented in
\texttt{constraint\_grammar.py}) shows:

\begin{itemize}
\item $\sigma_c = -2\pi/3$: the \textbf{only} admissible expression
  matching PDG to within $1\%$.
\item $\sigma_u = -\pi$: the \textbf{only} admissible expression
  matching PDG to within $1\%$.
\item $\sigma_b = 77/90$: wins over 4 other candidates by a factor
  of $40\times$ in error.
\end{itemize}

\noindent This is a \emph{negative result for the critic}: if the
grammar were rich enough to match anything, many candidates would
appear.  Instead, for the two angular quarks the grammar admits
exactly one candidate each.

%% --------------------------------------------------------------------
\subsection{PDG Scheme Pinning}
\label{sec:scheme-defense}

All quark mass comparisons use the standard PDG~2024 convention:
pole mass for top, $\overline{\mathrm{MS}}$ at $m_q(m_q)$ for charm
and bottom, $\overline{\mathrm{MS}}$ at $2$~GeV for light quarks.
Scheme sensitivity is documented in Supplement~VI, \S11 and
\texttt{pdg\_scheme\_pinning.py}.

\begin{itemize}
\item Top: model matches pole mass ($172.57$~GeV), \emph{not}
  $\overline{\mathrm{MS}}$ at $m_t$ ($\approx 162.5$~GeV).
\item Light quarks: model matches $2$~GeV convention; running to
  $1$~GeV shifts masses $\sim 20\%$, breaking the match.
\item The comparison is fully reproducible: every scheme choice and
  scale is documented.
\end{itemize}

\newpage
%% ====================================================================
\section{Critique-to-Response Map}
\label{sec:critique-map}
%% ====================================================================

The following table provides a one-stop reference: for every likely
critique, the specific test that addresses it, the result, and the
section where it is developed.

\begin{table}[ht]
\centering
\begin{tabular}{p{3.0cm}p{3.5cm}p{3.5cm}r}
\toprule
Critique & Test & Result & \S \\
\midrule
``Lucky coincidence''
  & Look-elsewhere MC
  & 0 hits in $10^5$ trials
  & \ref{sec:look-elsewhere} \\[4pt]
``Pipeline always works''
  & Negative controls
  & All 5 controls fail
  & \ref{sec:negative-controls} \\[4pt]
``Bug / artifact''
  & Independent reimplementation
  & Agreement $< 10^{-10}$
  & \ref{sec:reimplementation} \\[4pt]
``Cherry-picking''
  & Preregistered selection criteria
  & Criteria fixed before data
  & \ref{sec:forking-paths} \\[4pt]
``Floating-point''
  & High-precision verification
  & Exact arithmetic agrees
  & \ref{sec:reimplementation} \\[4pt]
``Data cherry-picking''
  & PDG pin + preregistration
  & Frozen constants \& config
  & \ref{sec:data-provenance} \\[4pt]
``Could map anything''
  & Permutation test
  & All permutations fail
  & \ref{sec:permutation} \\[4pt]
``Tuned the scan''
  & Analytic sieve $\eta = 2n/p^n$
  & No numerical scan needed
  & \ref{sec:reproducibility} \\[4pt]
``Mapping is optional''
  & Dictionary spec D1--D8
  & Machine-verified
  & \ref{sec:covering} \\
\bottomrule
\end{tabular}
\caption{Critique-to-response reference.  Every plausible objection has
a concrete, auditable test.}
\label{tab:critique-map}
\end{table}

\newpage
%% ====================================================================
\section{Reproducibility Protocol}
\label{sec:reproducibility}
%% ====================================================================

%% --------------------------------------------------------------------
\subsection{Search space}
\label{sec:search-space}

The parameter scan covers all pairs $(n, p)$ with
\begin{equation}
2 \leq n \leq 10, \qquad 2 \leq p \leq 30,
\end{equation}
yielding a raw candidate count of $9 \times 29 = 261$ pairs.

For each pair, the twist is computed analytically:
\begin{equation}
\eta(n, p) \;=\; \frac{2n}{p^n}\,.
\end{equation}

The Koide phase is
\begin{equation}
\delta(p) \;=\; \frac{2\pi}{p} + \eta(n, p),
\end{equation}
and the Koide parameter is $K_p = 2/p$.

%% --------------------------------------------------------------------
\subsection{Kill criteria (sequential)}
\label{sec:kill-criteria}

Candidates are eliminated in sequence.  A candidate must survive
all criteria to pass.

\begin{enumerate}[(K1)]
\item \textbf{Resonance lock.}  The phase $\delta$ must satisfy the
  resonance condition $\delta = 2\pi/p + \eta$, where $\eta$ is the
  spectral eta invariant.  (This is the defining equation, not a
  filter; it fixes $\delta$ given $(n, p)$.)

\item \textbf{Positive masses.}  All three eigenvalues of the Koide
  circulant must be positive.

\item \textbf{Non-degeneracy.}  The three eigenvalues must be distinct
  (i.e.\ the mass spectrum is non-degenerate).

\item \textbf{Physical viability.}  The predicted mass ratios must be
  within the range of known particle physics (no masses above the
  Planck scale, no negative masses, no tachyonic states).
\end{enumerate}

%% --------------------------------------------------------------------
\subsection{Elimination tally}

\begin{center}
\begin{tabular}{lr}
\toprule
Stage & Candidates remaining \\
\midrule
Raw pairs $(n, p)$ & 261 \\
After resonance lock (well-defined $\delta$) & 96 \\
After positive masses & 12 \\
After non-degeneracy & 4 \\
After physical viability & 1 \\
\midrule
\textbf{Survivor: $(n, p) = (3, 3)$} & \textbf{1} \\
\bottomrule
\end{tabular}
\end{center}

The sieve is analytic: no numerical optimisation, no gradient descent,
no fitting.  The formula $\eta = 2n/p^n$ produces a discrete set of
candidates, and structural constraints eliminate all but one.

%% --------------------------------------------------------------------
\subsection{Code listing}

The following scripts implement the full pipeline:
\begin{enumerate}[(i)]
\item \texttt{EtaInvariant.py} --- primary analysis pipeline: computes
  all 26 parameters from $(n, p) = (3, 3)$, performs the selection sieve,
  and outputs predictions with residuals.

\item \texttt{leng\_replication.py} --- independent reimplementation
  sharing no imports with the primary pipeline.  Used for the
  cross-validation in \S\ref{sec:reimplementation}.

\item \texttt{pytest} suite --- automated test suite verifying all
  predictions, residuals, correction coefficients, and negative
  controls.  Run with \texttt{pytest -v} from the repository root.
\end{enumerate}

All code is available in the repository and is version-controlled.

\newpage
%% ====================================================================
\section{Covering vs.\ Quotient: The Origin of Physical Content}
\label{sec:covering}
%% ====================================================================

\begin{theorem}[Origin of physical content]
All physical content of the framework is the spectral difference
between the covering space $S^5$ and its quotient $S^5/\mathbb{Z}_3$.
\end{theorem}

\begin{proof}[Structural argument]
On the covering space $S^5$, the $\ell = 1$ spherical harmonics are
valid eigenmodes of the Laplacian, with eigenvalue $\lambda_1 = 5$ and
degeneracy $d_1 = 6$.  On the quotient $S^5/\mathbb{Z}_3$, the
$\mathbb{Z}_3$ projection kills all six $\ell = 1$ modes (they carry
charges $\omega$ and $\omega^2$, not~$1$).

This spectral gap---present on the quotient, absent on the cover---is
the origin of all physical predictions:
\begin{itemize}
\item The \emph{lepton mass phase} is the asymmetry of what survives
  the projection (eta invariant $\eta = 2/9$).
\item The \emph{proton mass} is the spectral weight of what does not
  survive (the ghost gap, $d_{1,\mathrm{inv}} = 0$).
\item All 26~predictions are dimensionless ratios of spectral invariants
  of this single covering $\to$ quotient map.
\end{itemize}
\end{proof}

\begin{remark}[Nothing left to choose]
The covering $S^5$ is fixed: it is the unique simply-connected compact
manifold of dimension~5 with constant positive curvature.  The quotient
group $\mathbb{Z}_3$ is selected by the uniqueness argument
(Supplement~I, \S3).  The projection is determined by the group action.
Every spectral invariant is then computable.  There are no remaining
free parameters.
\end{remark}

\newpage
%% ====================================================================
\section{Methodological Notes}
\label{sec:methodology}
%% ====================================================================

%% --------------------------------------------------------------------
\subsection{Constraint-as-definition}
\label{sec:constraint-as-definition}

The equation $F(M) = 0$ is \emph{not} an equation between
independently sourced quantities.  It is a constraint that the geometry
either satisfies or does not.  The distinction matters: in conventional
physics, one tunes parameters until an equation is satisfied.  Here,
there are no parameters to tune.  A manifold either has
$\eta_D = 2/9$ or it does not.

\begin{remark}
This is why the framework has zero free parameters: the ``equation''
is really a definition.  The manifold is selected, not fitted.
\end{remark}

%% --------------------------------------------------------------------
\subsection{Three-as-one}
\label{sec:three-as-one}

The three lepton masses $(m_e, m_\mu, m_\tau)$ are \emph{not} three
independent quantities.  They are the three eigenvalues of a single
circulant matrix, determined by one geometric object: the toric fibre
of the orbifold.  Predicting all three masses from one phase~$\delta$
is therefore not ``three predictions''---it is one prediction with
three observable consequences.

%% --------------------------------------------------------------------
\subsection{Sieve by self-consistency}
\label{sec:sieve}

The uniqueness of $(n, p) = (3, 3)$ arises from the overlap of three
independent constraint systems:
\begin{enumerate}[(i)]
\item \textbf{Spectral geometry:} the eta invariant and degeneracy
  formulae of $S^{2n-1}/\mathbb{Z}_p$.
\item \textbf{Toric geometry:} the Koide circulant structure and the
  requirement $K = 2/p$.
\item \textbf{Number theory:} the twist formula $\eta = 2n/p^n$ and
  the requirement that $p$ be prime.
\end{enumerate}

Each system alone admits multiple solutions.  Their intersection
contains exactly one point: $(3, 3)$.

%% --------------------------------------------------------------------
\subsection{Binary quantum state}
\label{sec:binary}

The ghost mode (the $\ell = 1$ harmonic on the quotient) either exists
in the physical spectrum or does not.  There is no continuous parameter
controlling its presence.  The $L^2$ norm condition and the
$\mathbb{Z}_3$ projection together force
\begin{equation}
f_{\mathrm{on\text{-}shell}} \;=\; 1\,,
\end{equation}
meaning the mode is fully on-shell (exists with unit norm) or
identically zero.  This is a binary quantum state, not a continuous
variable.

\newpage
%% ====================================================================
\section{Falsification Thresholds}
\label{sec:falsification}
%% ====================================================================

A credible framework must be falsifiable.  The following table specifies
quantitative thresholds: if any measurement falls outside the stated
range, the framework is in tension or falsified.

\begin{table}[ht]
\centering
\begin{tabular}{p{2.5cm}p{2.5cm}p{2.0cm}p{4.5cm}}
\toprule
Observable & Prediction & Threshold & Falsification criterion \\
\midrule
$m_\tau$ (Belle~II)
  & 1776.985~MeV
  & $|\Delta| > 0.5$~MeV
  & Deviation exceeding $0.5$~MeV from predicted value \\[4pt]
4th generation
  & $N_g = 3$
  & Any detection
  & Discovery of any 4th-generation charged lepton \\[4pt]
Free quarks
  & $d_{\mathrm{inv}}(\ell\!=\!1) = 0$
  & Any detection
  & Observation of any isolated quark \\[4pt]
QCD axion
  & $\theta_{\mathrm{QCD}} = 0$
  & Any detection
  & Discovery of a QCD axion \\[4pt]
$\sum m_\nu$ (DESI)
  & 59.2~meV
  & $> 80$ or $< 40$~meV
  & Cosmological sum outside the window $[40, 80]$~meV \\[4pt]
$\alpha_s(M_Z)$
  & 0.1187
  & $> 3\sigma$ deviation
  & PDG world average deviating more than $3\sigma$ \\[4pt]
$G = 10/9$
  & Proton coeff.
  & Disagrees
  & Rigorous spectral calculation contradicts $G = 10/9$ \\[4pt]
$\sin^2\theta_W$
  & $3/8$ (GUT)
  & Threshold crossing
  & High-precision measurement inconsistent with $3/8$ at GUT scale \\[4pt]
$\delta_{\mathrm{CP}}$(PMNS)
  & Framework value
  & $> 5\sigma$
  & DUNE/HK measurement inconsistent at $> 5\sigma$ \\[4pt]
$\Delta m^2_{32}/\Delta m^2_{21}$
  & 33
  & $> 3\sigma$ from 33
  & Precision oscillation data inconsistent at $> 3\sigma$ \\
\bottomrule
\end{tabular}
\caption{Falsification thresholds.  Each row specifies the prediction,
the tolerance, and the criterion that would place the framework in
tension or falsify it outright.}
\label{tab:falsification}
\end{table}

\begin{remark}
Falsification is \emph{asymmetric}: a single clear violation of
$N_g = 3$ (discovery of a 4th generation) or $\theta_{\mathrm{QCD}} = 0$
(discovery of a QCD axion) would be immediately fatal.  Continuous
predictions like $m_\tau$ or $\alpha_s$ require threshold judgments
because of radiative corrections (Section~\ref{sec:hurricane}).
\end{remark}

\newpage
%% ====================================================================
%% BIBLIOGRAPHY
%% ====================================================================

\begin{thebibliography}{99}

\bibitem{pdg2024}
R.~L.~Workman \textit{et al.}\ (Particle Data Group),
``Review of Particle Physics,''
\textit{Prog.\ Theor.\ Exp.\ Phys.}\ \textbf{2024}, 083C01 (2024).

\bibitem{donnelly1978b}
H.~Donnelly,
``Spectrum and the Fixed Point Sets of Isometries~I,''
\textit{Math.\ Ann.}\ \textbf{224}, 161--170 (1978).

\bibitem{cheeger1979}
J.~Cheeger,
``Analytic Torsion and the Heat Equation,''
\textit{Ann.\ Math.}\ \textbf{109}, 259--322 (1979).

\bibitem{vafawitten1984}
C.~Vafa and E.~Witten,
``Parity Conservation in Quantum Chromodynamics,''
\textit{Phys.\ Rev.\ Lett.}\ \textbf{53}, 535--536 (1984).

\bibitem{connes1997}
A.~Connes,
``Noncommutative Geometry and the Standard Model,''
\textit{J.\ Math.\ Phys.}\ \textbf{38}, 1203--1208 (1997).

\bibitem{chamseddine2008}
A.~H.~Chamseddine and A.~Connes,
``Why the Standard Model,''
\textit{J.\ Geom.\ Phys.}\ \textbf{58}, 38--47 (2008).

\end{thebibliography}

\end{document}
