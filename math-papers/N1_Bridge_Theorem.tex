\documentclass[12pt]{article}
\usepackage{amsmath,amssymb,amsthm}
\usepackage{geometry}
\usepackage{hyperref}

\geometry{margin=1.2in}

\hypersetup{colorlinks=true, linkcolor=blue, citecolor=blue}

\newtheorem{theorem}{Theorem}
\newtheorem{lemma}[theorem]{Lemma}
\newtheorem{corollary}[theorem]{Corollary}
\newtheorem{proposition}[theorem]{Proposition}
\newtheorem{definition}[theorem]{Definition}
\newtheorem{remark}[theorem]{Remark}
\newtheorem{example}[theorem]{Example}

\title{\textbf{Cutoff Independence of Spectral Projections\\on Finite Group Quotients}\\[0.5em]
\large The $N=1$ Bridge Theorem}

\author{Jixiang Leng}
\date{February 2026}

\begin{document}
\maketitle

\begin{abstract}
Let $D$ be a self-adjoint operator with discrete spectrum on a Hilbert space $\mathcal{H}$,
and let $G$ be a finite group acting on $\mathcal{H}$ by isometries that commute with $D$.
We prove that for any bounded Borel function $f$ and any minimal central idempotent $e_m$
of the group algebra $\mathbb{C}[G]$, the trace $\mathrm{Tr}(f(D)\,e_m)$ equals
$\mathrm{Tr}(f(D_m))$ where $D_m = D|_{e_m\mathcal{H}}$, with coefficient exactly $1$.
In particular, the spectral correction from each group sector is independent of the
choice of cutoff function $f$.  Applied to the Dirac operator on $S^5/\mathbb{Z}_3$,
this eliminates all cutoff-function ambiguity in the Yukawa phase correction and
forces $N=1$ in the spectral action framework.
\end{abstract}

%% ============================================================
\section{Introduction}
%% ============================================================

In the spectral action approach to particle physics~\cite{connes1996,chamseddine2011},
the bosonic action is extracted from the trace $\mathrm{Tr}(f(D/\Lambda))$ where $D$
is the Dirac operator and $f$ is a cutoff function.  Physical predictions that depend
on the \emph{choice} of $f$ are not genuine predictions but artifacts of the regularization.

On an orbifold $M/G$, the Dirac spectrum decomposes into sectors labeled by the
irreducible representations of $G$.  The spectral correction in each sector involves
a coefficient $N$ multiplying the eta invariant.  If $N$ depends on $f$, the correction
is scheme-dependent and therefore unphysical.  If $N$ is independent of $f$, the
correction is a geometric invariant.

We prove that $N = 1$ exactly, for any $f$, as a consequence of the commutativity
$[f(D), e_m] = 0$ when $G$ acts by isometries.  The proof uses only functional calculus
and the spectral theorem; no special properties of $f$ beyond measurability are required.

%% ============================================================
\section{Setup and Notation}
%% ============================================================

\begin{definition}[Spectral triple with finite group action]\label{def:setup}
Let $(\mathcal{H}, D)$ be a spectral datum consisting of:
\begin{itemize}
\item A separable Hilbert space $\mathcal{H}$.
\item A self-adjoint operator $D$ with compact resolvent (hence discrete spectrum
$\{\lambda_n\}_{n \in \mathbb{N}}$ with $|\lambda_n| \to \infty$).
\end{itemize}
Let $G$ be a finite group acting on $\mathcal{H}$ by unitary operators
$\rho: G \to \mathcal{U}(\mathcal{H})$ satisfying the \textbf{isometry condition}:
\begin{equation}
\rho(g)\, D = D\, \rho(g) \qquad \text{for all } g \in G.
\label{eq:commutation}
\end{equation}
\end{definition}

\begin{definition}[Minimal central idempotents]\label{def:idempotents}
The group algebra $\mathbb{C}[G]$ decomposes as a direct sum of matrix algebras,
one for each irreducible representation $\pi_m$ of $G$ ($m = 0, 1, \ldots, r-1$
where $r$ is the number of irreducible representations).  The \textbf{minimal central
idempotents} are:
\begin{equation}
e_m = \frac{\dim \pi_m}{|G|} \sum_{g \in G} \overline{\chi_m(g)}\, \rho(g),
\label{eq:idempotent}
\end{equation}
where $\chi_m$ is the character of $\pi_m$.  These satisfy:
\begin{align}
e_m^2 &= e_m, \label{eq:idem-square}\\
e_m e_{m'} &= 0 \quad\text{for } m \neq m', \label{eq:idem-orthogonal}\\
\sum_{m=0}^{r-1} e_m &= \mathbf{1}. \label{eq:partition-of-unity}
\end{align}
\end{definition}

\begin{definition}[Sector Hilbert space and restricted operator]
The $m$-th sector Hilbert space is $\mathcal{H}_m = e_m \mathcal{H}$.
The restricted operator is $D_m = D|_{\mathcal{H}_m}$.
\end{definition}

%% ============================================================
\section{The Main Theorem}
%% ============================================================

\begin{theorem}[$N=1$: Cutoff Independence]\label{thm:N1}
Let $(\mathcal{H}, D, G, \rho)$ be as in Definition~\ref{def:setup}, satisfying the
isometry condition~\eqref{eq:commutation}.  Let $f: \mathbb{R} \to \mathbb{C}$ be a Borel function such that $f(D)$ is
trace-class\footnote{This holds for heat kernels $f(x) = e^{-tx^2}$,
resolvents $f(x) = (x^2 + m^2)^{-s}$ with $\mathrm{Re}(s)$ sufficiently large,
smooth cutoffs with sufficient decay, and more generally any $f$ for which
$\sum_n |f(\lambda_n)| < \infty$.  The eta function case $f(x) = \mathrm{sign}(x)|x|^{-s}$
is handled by analytic continuation from the region of convergence, not as a literal
bounded function.}.
Then for each minimal central idempotent $e_m$:
\begin{equation}
\boxed{\mathrm{Tr}(f(D)\, e_m) = \mathrm{Tr}(f(D_m)).}
\label{eq:main}
\end{equation}
In particular, if $\eta_m = \sum_{\lambda \in \mathrm{spec}(D_m)} \mathrm{sign}(\lambda)\,|\lambda|^{-s}|_{s=0}$
is the eta invariant of the $m$-th sector, then the coefficient of $\eta_m$ in any
spectral action trace is exactly $1$, independent of $f$.
\end{theorem}

\begin{proof}
The proof proceeds in three steps.

\medskip\noindent\textbf{Step 1: $f(D)$ commutes with $e_m$.}

Since $G$ acts by isometries, $\rho(g) D = D \rho(g)$ for all $g \in G$.
By the functional calculus for self-adjoint operators, for any bounded Borel function $f$:
\begin{equation}
\rho(g)\, f(D) = f(D)\, \rho(g) \qquad \text{for all } g \in G.
\label{eq:f-commutes-g}
\end{equation}
\emph{Proof of~\eqref{eq:f-commutes-g}:}
Let $D = \int \lambda\, dE(\lambda)$ be the spectral decomposition, so
$f(D) = \int f(\lambda)\, dE(\lambda)$.  Since $\rho(g)$ commutes with $D$,
it commutes with every spectral projection $E(B)$ for Borel sets $B \subset \mathbb{R}$
(standard result: see~\cite{reed-simon}, Theorem VIII.5).  Therefore $\rho(g)$ commutes
with $\int f(\lambda)\, dE(\lambda) = f(D)$.

Since $e_m$ is a $\mathbb{C}$-linear combination of the $\rho(g)$
(equation~\eqref{eq:idempotent}), it follows that:
\begin{equation}
[f(D),\, e_m] = 0.
\label{eq:f-commutes-em}
\end{equation}

\medskip\noindent\textbf{Step 2: The trace decomposes.}

Since $e_m$ is a projection ($e_m^2 = e_m$) and commutes with $f(D)$,
the operator $f(D)\, e_m = e_m\, f(D)\, e_m$ acts on $\mathcal{H}_m = e_m\mathcal{H}$.
Let $\{|\psi_n^{(m)}\rangle\}$ be an orthonormal eigenbasis of $D_m$ in $\mathcal{H}_m$
with eigenvalues $\{\lambda_n^{(m)}\}$.  Then:
\begin{align}
\mathrm{Tr}(f(D)\, e_m)
&= \sum_n \langle \psi_n^{(m)} | f(D)\, e_m | \psi_n^{(m)} \rangle
\label{eq:trace-expand}\\
&= \sum_n \langle \psi_n^{(m)} | f(D) | \psi_n^{(m)} \rangle
\qquad \text{(since $e_m |\psi_n^{(m)}\rangle = |\psi_n^{(m)}\rangle$)}
\label{eq:em-acts}\\
&= \sum_n f(\lambda_n^{(m)})
\qquad \text{(since $|\psi_n^{(m)}\rangle$ is an eigenstate of $D$ with eigenvalue $\lambda_n^{(m)}$)}
\label{eq:eigenvalue}\\
&= \mathrm{Tr}(f(D_m)).
\label{eq:final}
\end{align}

The key step is~\eqref{eq:eigenvalue}: $|\psi_n^{(m)}\rangle \in \mathcal{H}_m$ is an
eigenvector of $D_m = D|_{\mathcal{H}_m}$, and since $[D, e_m] = 0$
(which follows from~\eqref{eq:commutation} and the definition of $e_m$),
it is also an eigenvector of $D$ with the \emph{same} eigenvalue.
Therefore $f(D)|\psi_n^{(m)}\rangle = f(\lambda_n^{(m)})|\psi_n^{(m)}\rangle$.

\medskip\noindent\textbf{Step 3: The coefficient is $1$.}

Equation~\eqref{eq:final} shows $\mathrm{Tr}(f(D)\, e_m) = \mathrm{Tr}(f(D_m))$
with coefficient exactly $1$.  This holds for \emph{any} bounded Borel $f$,
including:
\begin{itemize}
\item $f(x) = e^{-tx^2}$ (heat kernel; trace-class for $t > 0$),
\item $f(x) = (x^2 + m^2)^{-s}$ (resolvent; trace-class for $\mathrm{Re}(s) > \dim(M)/2$),
\item Any Schwartz-class function (smooth cutoff; trace-class by rapid decay),
\item The eta function $\eta_m(s) = \mathrm{Tr}(\mathrm{sign}(D_m)|D_m|^{-s})$
via analytic continuation from $\mathrm{Re}(s) \gg 0$.
\end{itemize}
In particular, if we define the spectral correction as
$\Delta_m = N \cdot \eta_m$ where $\eta_m$ is the eta invariant of the $m$-th sector,
then $N = 1$ independently of the regularization scheme.  \qed
\end{proof}

%% ============================================================
\section{Application to $S^5/\mathbb{Z}_3$}
%% ============================================================

\begin{corollary}[Yukawa phase on $S^5/\mathbb{Z}_3$]\label{cor:yukawa}
Let $D$ be the Dirac operator on $S^5$ equipped with the round metric of radius $R$.
Let $\mathbb{Z}_3$ act by $z_j \mapsto \omega z_j$ ($\omega = e^{2\pi i/3}$,
$j = 1,2,3$) on $S^5 \subset \mathbb{C}^3$.  This action commutes with $D$
(since it acts by isometries of the round metric).

The group $\mathbb{Z}_3$ has three irreducible representations:
$\chi_0$ (trivial), $\chi_1$ (character $\omega$), $\chi_2$ (character $\omega^2$).
The corresponding minimal idempotents are
$e_m = \frac{1}{3}\sum_{k=0}^{2} \omega^{-mk} \rho(\omega^k)$.

By Theorem~\ref{thm:N1}, for any cutoff function $f$:
\begin{equation}
\mathrm{Tr}(f(D)\, e_m) = \mathrm{Tr}(f(D_m)) = \sum_{\lambda \in \mathrm{spec}(D_m)} f(\lambda).
\end{equation}

The Donnelly eta invariant of the $m$-th sector is
$\eta_D(\chi_m) = \pm i/9$ (computed in~\cite{donnelly1978}).
By the theorem, the coefficient of $\eta_D(\chi_m)$ in the Yukawa phase correction
is exactly $1$, regardless of the cutoff function used in the spectral action.
\end{corollary}

\begin{proof}
The $\mathbb{Z}_3$ action on $S^5$ is by isometries (rotation by $2\pi/3$ in each
complex coordinate plane preserves the round metric).  Therefore
$\rho(g) D = D \rho(g)$ for all $g \in \mathbb{Z}_3$.
Theorem~\ref{thm:N1} applies with $G = \mathbb{Z}_3$.  \qed
\end{proof}

\begin{remark}[Significance for the spectral action]
In the Connes--Chamseddine spectral action~\cite{connes1996,chamseddine2011},
the bosonic action is $\mathrm{Tr}(f(D/\Lambda))$ and physical predictions arise
from the Seeley--DeWitt expansion.  The choice of cutoff function $f$ affects
individual Seeley--DeWitt coefficients through the ``moments'' $f_k = \int_0^\infty f(u)\, u^{k/2-1}\, du$.
The $N=1$ theorem guarantees that the sector decomposition --- how the spectral
content distributes among $\mathbb{Z}_3$ representations --- is \emph{independent}
of these moments.  The eta invariant contributes $\eta_D(\chi_m)$ to the $m$-th sector
with coefficient exactly $1$, regardless of whether $f$ is a sharp cutoff, a smooth
cutoff, or any other admissible function.

This eliminates a potential source of scheme dependence that would otherwise
undermine the derivation of the Koide phase $\delta = 2\pi/3 + 2/9$ from the spectral
geometry of $S^5/\mathbb{Z}_3$.
\end{remark}

%% ============================================================
\section{Generalization}
%% ============================================================

\begin{remark}[Generality of the result]
Theorem~\ref{thm:N1} holds for:
\begin{enumerate}
\item Any finite group $G$ (not just cyclic groups).
\item Any self-adjoint operator with compact resolvent (not just the Dirac operator).
\item Any bounded Borel function $f$ (not just smooth or Schwartz-class cutoffs).
\item Any Riemannian manifold $M$ on which $G$ acts by isometries.
\end{enumerate}
The only requirement is the isometry condition~\eqref{eq:commutation}: the group action
must commute with the operator.  For the Dirac operator on a Riemannian manifold,
this is equivalent to requiring that $G$ act by isometries --- a geometric condition,
not an analytic one.
\end{remark}

\begin{remark}[What fails without the isometry condition]
If $G$ does \emph{not} act by isometries (e.g., if the group action deforms the metric),
then $[D, \rho(g)] \neq 0$ and the functional calculus argument in Step~1 fails.
In that case, $[f(D), e_m] \neq 0$ for generic $f$, and the coefficient $N$ depends
on the choice of cutoff function.  The $N=1$ result is therefore a genuine theorem
about isometric group actions, not a tautology.
\end{remark}

%% ============================================================
\section{Summary}
%% ============================================================

The $N=1$ bridge theorem states that on any orbifold $M/G$ where $G$ acts by isometries,
the decomposition of the spectral action into group-representation sectors is independent
of the cutoff function.  The proof is three lines of functional calculus:
(1)~isometry implies $[\rho(g), D] = 0$;
(2)~functional calculus gives $[\rho(g), f(D)] = 0$ for any Borel $f$;
(3)~$e_m$ is a linear combination of $\rho(g)$, so $[f(D), e_m] = 0$,
and the trace $\mathrm{Tr}(f(D)\, e_m) = \mathrm{Tr}(f(D_m))$ with coefficient $1$.

Applied to $S^5/\mathbb{Z}_3$: the Yukawa phase correction from the Donnelly eta invariant
is exactly $\eta_D(\chi_m)$ per sector, with no cutoff ambiguity.  This is the mathematical
foundation for the scheme-independent derivation of the Koide mass ratios from spectral geometry.

%% ============================================================
\begin{thebibliography}{99}

\bibitem{connes1996}
A.~Connes, ``Gravity coupled with matter and the foundation of non-commutative geometry,''
\textit{Commun.\ Math.\ Phys.}\ \textbf{182} (1996) 155.

\bibitem{chamseddine2011}
A.~H.~Chamseddine and A.~Connes, ``The uncanny precision of the spectral action,''
\textit{Commun.\ Math.\ Phys.}\ \textbf{307} (2011) 735.

\bibitem{donnelly1978}
H.~Donnelly, ``Eta invariants for $G$-spaces,''
\textit{Indiana Univ.\ Math.\ J.}\ \textbf{27} (1978) 889--918.

\bibitem{reed-simon}
M.~Reed and B.~Simon, \textit{Methods of Modern Mathematical Physics},
Vol.~I: Functional Analysis, Academic Press, 1980.

\end{thebibliography}

\end{document}
