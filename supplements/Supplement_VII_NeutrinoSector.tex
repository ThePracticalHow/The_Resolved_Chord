\documentclass[12pt]{article}
\usepackage{amsmath,amssymb,amsthm}
\usepackage{geometry}
\usepackage{booktabs}
\usepackage{parskip}
\usepackage{enumerate}
\usepackage{microtype}

\geometry{margin=1.2in}
\emergencystretch=1em

\newtheorem{theorem}{Theorem}
\newtheorem{corollary}{Corollary}
\newtheorem{proposition}{Proposition}
\newtheorem{definition}{Definition}
\newtheorem{remark}{Remark}
\newtheorem{lemma}{Lemma}

\title{\textbf{Supplement VII: The Neutrino Sector --- Parameters 21--26}\\[0.3em]
\large Complete Derivation Chain for the Neutrino Mixing and Mass Sector\\[0.2em]
\normalsize The Resolved Chord --- Supplementary Material}

\author{Jixiang Leng}
\date{February 2026}

\begin{document}
\maketitle

\begin{abstract}
This supplement is self-contained.  It provides the complete derivation chain
for Parameters~21--26 of the main text: the three PMNS mixing angles
(reactor, solar, atmospheric), the leptonic CP phase, the heaviest neutrino
mass, and the mass-squared splitting ratio.  The key structural distinction
--- twisted versus untwisted sectors of the $\mathbb{Z}_3$ orbifold ---
is developed first, followed by individual parameter derivations, the unified
point/side/face forward model, and derived predictions for cosmological
observables.  All definitions, intermediate calculations, and numerical
verifications are included.
\end{abstract}

\tableofcontents
\newpage

%% ====================================================================
\section{Twisted versus Untwisted Sectors}
\label{sec:twisted-untwisted}
%% ====================================================================

The $\mathbb{Z}_3$ orbifold $S^5/\mathbb{Z}_3$ divides the five-sphere into
three $120^{\circ}$ sectors.  Two qualitatively distinct classes of geometric
structure arise: \emph{fold walls} (the smooth codimension-1 interfaces
separating adjacent sectors) and the \emph{cone point} (the singular
fixed-point locus carrying a deficit angle of $2\pi/3$).  These structures
govern the physics of the two fermion sectors.

%% --------------------------------------------------------------------
\subsection{Twisted sector: charged fermions}
\label{sec:twisted}

\begin{definition}[Twisted-sector string]
A string on $S^5/\mathbb{Z}_3$ is \emph{twisted} if it is closed only
modulo the $\mathbb{Z}_3$ action: its endpoint returns to its starting
point only after application of the generator $g$.
\end{definition}

Twisted-sector strings are \emph{pinned to the cone point}, the singular
locus of the orbifold.  They carry a twist phase $\omega^k$
($k = 1, 2$; $\omega = e^{2\pi i/3}$) encoding the $\mathbb{Z}_3$ charge.
The key consequences are:

\begin{enumerate}[(i)]
  \item \textbf{Mass origin.}  Masses of twisted-sector fermions arise from
    the topological structure of the conical defect.  The deficit angle
    $2\pi/3$ sets the harmonic content of the wavefunctions localised at
    the cone tip.

  \item \textbf{Koide ratio.}  All three charged-lepton masses are
    eigenvalues of the \emph{same} circulant mass matrix (Supplement~II,
    \S1), determined by the single topological invariant of the cone point.
    The Koide ratio
    \begin{equation}
      K \;=\; \frac{m_e + m_\mu + m_\tau}
        {(\sqrt{m_e}+\sqrt{m_\mu}+\sqrt{m_\tau})^2}
      \;=\; \frac{2}{3} \quad\text{(exact)}
    \end{equation}
    is a consequence of the circulant structure: it measures the harmonic
    content of a fixed topological feature (the cone point).

  \item \textbf{CKM mixing.}  Inter-generation mixing among quarks
    (also twisted-sector fermions) requires tunnelling \emph{through} the
    singular cone point.  This tunnelling amplitude is exponentially
    suppressed by the conical geometry, producing the observed smallness
    of off-diagonal CKM elements.
\end{enumerate}

%% --------------------------------------------------------------------
\subsection{Untwisted sector: neutrinos}
\label{sec:untwisted}

\begin{definition}[Untwisted-sector string]
A string on $S^5/\mathbb{Z}_3$ is \emph{untwisted} if it is closed
without application of the $\mathbb{Z}_3$ generator: it satisfies
standard periodicity on the quotient space.
\end{definition}

Untwisted-sector strings are \emph{not} pinned to the cone point.  They
propagate freely through the bulk of each $120^{\circ}$ sector and interact
with the \emph{fold walls} --- the smooth, extended, codimension-1 interfaces
between adjacent sectors.  The consequences are sharply distinct from the
twisted sector:

\begin{enumerate}[(i)]
  \item \textbf{Mass origin.}  Neutrino masses arise from tunnelling overlaps
    between wavefunctions localised at different fold walls.  Because the
    fold walls are smooth extended surfaces (not singular points), these
    overlaps are geometric rather than topological.

  \item \textbf{Mixing angles.}  Tunnelling amplitudes between fold walls
    are of order one --- there is no exponential suppression from a singular
    barrier.  This produces the large mixing angles observed in the PMNS
    matrix.

  \item \textbf{Koide ratio.}  The three neutrino mass eigenstates are
    \emph{not} three copies of a single object with different $\mathbb{Z}_3$
    charges; they are three geometrically distinct objects (see
    Section~\ref{sec:PSF}).  The circulant symmetry that enforces $K = 2/3$
    for charged leptons does not apply, and the neutrino Koide-like ratio
    $Q_\nu \neq 2/3$ (see Section~\ref{sec:noKoide}).
\end{enumerate}

%% --------------------------------------------------------------------
\subsection{Summary comparison}
\label{sec:comparison}

\begin{table}[ht]
\centering
\begin{tabular}{@{} l l l @{}}
\toprule
\textbf{Property} & \textbf{Charged fermions (twisted)} & \textbf{Neutrinos (untwisted)} \\
\midrule
Pinned to          & Cone point (singular)             & Fold walls (smooth)          \\
Mass origin        & Topological (deficit angle)       & Geometric (tunnelling overlap) \\
Koide ratio        & $K = 2/3$ (exact)                 & $Q_\nu \neq 2/3$            \\
Mixing angles      & Small (exponential suppression)   & Large (order-1 tunnelling)   \\
\bottomrule
\end{tabular}
\caption{Twisted versus untwisted sectors of the $\mathbb{Z}_3$ orbifold.}
\label{tab:twisted-untwisted}
\end{table}

\begin{remark}
The twisted/untwisted distinction is not a model-building choice; it is
forced by the topology of $S^5/\mathbb{Z}_3$.  Electrically charged fermions
carry non-trivial $\mathbb{Z}_3$ representations and are therefore twisted;
neutrinos are neutral under the $\mathbb{Z}_3$ and are therefore untwisted.
The qualitative differences in Table~\ref{tab:twisted-untwisted} ---
small versus large mixing, exact versus approximate Koide --- follow as
geometric consequences.
\end{remark}

%% ====================================================================
\section{Tribimaximal Mixing as Zeroth Order}
\label{sec:TBM}
%% ====================================================================

The zeroth-order prediction for the PMNS matrix is tribimaximal mixing (TBM),
which arises as a \emph{mathematical identity} from the $\mathbb{Z}_3$
symmetry.

\subsection{$\mathbb{Z}_3$ permutation eigenvectors}

The $\mathbb{Z}_3$ cyclic permutation operator on three objects is
\begin{equation}
  P \;=\;
  \begin{pmatrix} 0 & 0 & 1 \\ 1 & 0 & 0 \\ 0 & 1 & 0 \end{pmatrix},
  \qquad P^3 = I.
\end{equation}
Its eigenvalues are $\{1,\, \omega,\, \omega^2\}$ with
$\omega = e^{2\pi i/3}$.

\begin{proposition}[TBM from $\mathbb{Z}_3$]
\label{prop:TBM}
The eigenvectors of $P$ are precisely the columns of the tribimaximal
mixing matrix $U_{\mathrm{TBM}}$.
\end{proposition}

\begin{proof}
The normalised eigenvectors of $P$ are:
\begin{align}
  v_1 &= \frac{1}{\sqrt{3}}(1, 1, 1)^T
    &\quad&\text{(eigenvalue $1$)}, \\
  v_2 &= \frac{1}{\sqrt{3}}(1, \omega, \omega^2)^T
    &\quad&\text{(eigenvalue $\omega$)}, \\
  v_3 &= \frac{1}{\sqrt{3}}(1, \omega^2, \omega)^T
    &\quad&\text{(eigenvalue $\omega^2$)}.
\end{align}
The democratic matrix $D = \frac{1}{3}\mathbf{1}\mathbf{1}^T$ (all entries
$1/3$) commutes with $P$ and shares the same eigenvector structure.
Restricting to real combinations and imposing the conventional phase choices,
the resulting unitary matrix is
\begin{equation}
  U_{\mathrm{TBM}} \;=\;
  \begin{pmatrix}
    \sqrt{2/3} & 1/\sqrt{3} & 0 \\
    -1/\sqrt{6} & 1/\sqrt{3} & 1/\sqrt{2} \\
    1/\sqrt{6} & 1/\sqrt{3} & -1/\sqrt{2}
  \end{pmatrix},
\end{equation}
which yields the TBM mixing angles:
\begin{equation}
\label{eq:TBM-angles}
  \sin^2\theta_{12} = \frac{1}{3}, \qquad
  \sin^2\theta_{23} = \frac{1}{2}, \qquad
  \theta_{13} = 0.
\end{equation}
\end{proof}

\begin{remark}
Tribimaximal mixing is the \emph{undeformed} $\mathbb{Z}_3$ prediction.
Each measured deviation from the TBM values~\eqref{eq:TBM-angles} maps to
a specific geometric property of the physical orbifold:
\begin{itemize}
  \item $\theta_{13} \neq 0$: junction asymmetry (Section~\ref{sec:P21}),
  \item $\sin^2\theta_{12} \neq 1/3$: finite fold-wall thickness
    (Section~\ref{sec:P22}),
  \item $\sin^2\theta_{23} \neq 1/2$: spectral impedance mismatch
    (Section~\ref{sec:P23}).
\end{itemize}
\end{remark}

%% ====================================================================
\section{Parameter 21 --- Reactor Angle $\theta_{13}$}
\label{sec:P21}
%% ====================================================================

\subsection{The formula}

\begin{theorem}[Reactor angle]
\label{thm:reactor}
\begin{equation}
\label{eq:reactor}
  \boxed{\;
  \sin^2\theta_{13}
    \;=\; (\eta\, K)^2
    \;=\; \left(\frac{2}{9}\cdot\frac{2}{3}\right)^{\!2}
    \;=\; \left(\frac{4}{27}\right)^{\!2}
    \;=\; \frac{16}{729}.
  \;}
\end{equation}
\end{theorem}

\subsection{Numerical comparison}

\begin{equation}
  \sin^2\theta_{13}\big|_{\mathrm{pred}} = \frac{16}{729} = 0.02194.
\end{equation}
The measured value (NuFIT~5.3, normal ordering~\cite{nufit2024}):
\begin{equation}
  \sin^2\theta_{13}\big|_{\mathrm{meas}} = 0.02200 \pm 0.00069.
\end{equation}
\begin{equation}
  \text{Deviation:}\quad
  \frac{|0.02194 - 0.02200|}{0.02200} = 0.27\%.
\end{equation}

\subsection{Geometric derivation}

In the TBM limit (Section~\ref{sec:TBM}), exact $\mathbb{Z}_3$ symmetry
gives $\theta_{13} = 0$.  The physical orbifold $S^5/\mathbb{Z}_3$ breaks
this through two effects:

\begin{enumerate}[(i)]
  \item \textbf{Fold-wall bleed ($\eta = 2/9$).}
    The fold walls separating the three $120^{\circ}$ sectors have finite
    thickness.  The Donnelly invariant $\eta = 2/9$
    (Supplement~I, \S2; originally
    \cite{donnelly1978}) measures the spectral asymmetry induced by this
    finite thickness.  It is the leading-order correction to the perfect
    $\mathbb{Z}_3$ geometry.

  \item \textbf{Harmonic lock ($K = 2/3$).}
    The Koide constant $K = 2/3$ is the moment-map constraint from the
    orbifold harmonic decomposition (Supplement~V, \S3.3).  It enters
    because the reactor angle couples the twisted-sector harmonic structure
    to the untwisted-sector fold-wall geometry.
\end{enumerate}

The product $\eta \cdot K = (2/9)(2/3) = 4/27$ is the leading correction to
$\theta_{13} = 0$.  The result enters \emph{squared} because $\theta_{13}$
couples the first and third generations, which requires traversing
\emph{two} fold-wall transitions (from sector~1 across the intervening
sector to sector~3).

\begin{proposition}[Squaring from double transition]
\label{prop:squaring}
The mixing element $|U_{e3}|^2 = \sin^2\theta_{13}$ involves a
first-to-third generation transition.  In the $\mathbb{Z}_3$ geometry,
this requires two sequential fold-wall hops:
$\nu_1 \to \text{fold wall} \to \nu_2 \to \text{fold wall} \to \nu_3$.
Each hop contributes a factor of $\eta\, K$, giving the square
$(\eta\, K)^2$.
\end{proposition}

%% ====================================================================
\section{Parameter 22 --- Solar Angle $\theta_{12}$}
\label{sec:P22}
%% ====================================================================

\subsection{The formula}

\begin{theorem}[Solar angle]
\label{thm:solar}
\begin{equation}
\label{eq:solar}
  \boxed{\;
  \sin^2\theta_{12}
    \;=\; \frac{1}{3} - \frac{\eta^2}{2}
    \;=\; \frac{1}{3} - \frac{1}{2}\left(\frac{2}{9}\right)^{\!2}
    \;=\; \frac{1}{3} - \frac{2}{81}
    \;=\; \frac{27 - 2}{81}
    \;=\; \frac{25}{81}.
  \;}
\end{equation}
\end{theorem}

\subsection{Numerical comparison}

\begin{equation}
  \sin^2\theta_{12}\big|_{\mathrm{pred}} = \frac{25}{81} = 0.3086.
\end{equation}
The measured value (NuFIT~5.3~\cite{nufit2024}):
\begin{equation}
  \sin^2\theta_{12}\big|_{\mathrm{meas}} = 0.307 \pm 0.013.
\end{equation}
\begin{equation}
  \text{Deviation:}\quad
  \frac{|0.3086 - 0.307|}{0.307} = 0.53\%.
\end{equation}

\subsection{Geometric derivation}

The democratic $\mathbb{Z}_3$ symmetry gives $\sin^2\theta_{12} = 1/3$
(TBM value).  The correction arises from the finite thickness of the fold
walls:

\begin{enumerate}[(i)]
  \item The Donnelly invariant $\eta = 2/9$ measures the spectral asymmetry
    due to finite fold-wall thickness.  The correction to the solar angle
    is second-order in $\eta$ because $\theta_{12}$ connects the first and
    second generations, which are \emph{adjacent} sectors --- a single
    fold-wall transition.  The leading correction is therefore $\sim\eta^2$.

  \item The factor $1/2$ arises from the two-body nature of the 1--2
    overlap: the tunnelling amplitude between two adjacent fold walls
    involves a symmetric two-state system, introducing a factor of $1/2$
    in the perturbation expansion.

  \item The correction is \emph{negative} (reducing $\sin^2\theta_{12}$
    below $1/3$) because the finite fold-wall thickness partially
    localises the neutrino wavefunctions, reducing the overlap between
    the $\nu_1$ and $\nu_2$ states.
\end{enumerate}

\begin{remark}[Perfect square]
The result $25/81 = (5/9)^2$ is a perfect square of rationals.
This is not a coincidence: $5 = \lambda_1$ (the first non-trivial
eigenvalue on $S^5$) and $9 = 3^2 = p^2$, so the solar angle encodes
the ratio of the spectral gap to the square of the orbifold order.
\end{remark}

%% ====================================================================
\section{Parameter 23 --- Atmospheric Angle $\theta_{23}$}
\label{sec:P23}
%% ====================================================================

\subsection{The formula}

\begin{theorem}[Atmospheric angle]
\label{thm:atmospheric}
\begin{equation}
\label{eq:atmospheric}
  \boxed{\;
  \sin^2\theta_{23}
    \;=\; \frac{d_1}{d_1 + \lambda_1}
    \;=\; \frac{6}{6 + 5}
    \;=\; \frac{6}{11}.
  \;}
\end{equation}
\end{theorem}

\subsection{Numerical comparison}

\begin{equation}
  \sin^2\theta_{23}\big|_{\mathrm{pred}} = \frac{6}{11} = 0.5455.
\end{equation}
The measured value (NuFIT~5.3~\cite{nufit2024}):
\begin{equation}
  \sin^2\theta_{23}\big|_{\mathrm{meas}} = 0.546 \pm 0.021.
\end{equation}
\begin{equation}
  \text{Deviation:}\quad
  \frac{|0.5455 - 0.546|}{0.546} = 0.10\%.
\end{equation}

\subsection{Geometric derivation}

The TBM value $\sin^2\theta_{23} = 1/2$ corresponds to maximal 2--3 mixing.
The physical orbifold deviates from maximality through a spectral impedance
ratio at the fold interface:

\begin{enumerate}[(i)]
  \item \textbf{Available modes ($d_1 = 6$).}  The $\ell = 1$ eigenspace
    on $S^5$ has degeneracy $d_1 = 6$ (Supplement~V, \S1.1).  These six
    modes constitute the tunnelling bandwidth --- the number of channels
    available for fold-wall transmission.

  \item \textbf{Eigenvalue gap ($\lambda_1 = 5$).}  The $\ell = 1$
    eigenvalue $\lambda_1 = \ell(\ell+4)|_{\ell=1} = 5$ acts as a spectral
    barrier.  Modes with energy below $\lambda_1$ are reflected; those
    above are transmitted.

  \item \textbf{Impedance ratio.}  At the fold interface, the fraction of
    spectral weight carried by the fold modes (transmitted) versus
    reflected by the spectral gap is
    \begin{equation}
      \sigma \;=\; \frac{d_1}{d_1 + \lambda_1} \;=\; \frac{6}{11}.
    \end{equation}
    This ratio directly determines $\sin^2\theta_{23}$.
\end{enumerate}

\begin{remark}
The atmospheric angle is \emph{not} maximal ($1/2$) but close to it.
The deviation $6/11 - 1/2 = 1/22 \approx 0.045$ is a direct measure of
the spectral impedance mismatch between the degeneracy and eigenvalue at
$\ell = 1$.  Current experiments are approaching the precision needed to
confirm or exclude maximality; the prediction $6/11$ is distinguishable
from $1/2$ at the $\sim 2\sigma$ level with projected NOvA and T2K
sensitivities.
\end{remark}

%% ====================================================================
\section{Parameter 24 --- Leptonic CP Phase $\delta_{\mathrm{CP}}^{\mathrm{PMNS}}$}
\label{sec:P24}
%% ====================================================================

\subsection{The formula}

\begin{theorem}[Leptonic CP phase]
\label{thm:deltaCP}
\begin{equation}
\label{eq:deltaCP}
  \boxed{\;
  \delta_{\mathrm{CP}}^{\mathrm{PMNS}}
    \;=\; p\,\gamma
    \;=\; 3\,\arctan\!\left(\frac{2\pi^2}{9}\right)
    \;=\; 3 \times 65.5^{\circ}
    \;=\; 196.5^{\circ}.
  \;}
\end{equation}
\end{theorem}

\subsection{Numerical comparison}

\begin{equation}
  \delta_{\mathrm{CP}}^{\mathrm{PMNS}}\big|_{\mathrm{pred}} = 196.5^{\circ}.
\end{equation}
The measured value (combined T2K/NOvA, normal ordering~\cite{nufit2024}):
\begin{equation}
  \delta_{\mathrm{CP}}^{\mathrm{PMNS}}\big|_{\mathrm{meas}}
    = 195^{\circ} \pm 50^{\circ}.
\end{equation}
\begin{equation}
  \text{Deviation:}\quad
  \frac{|196.5 - 195|}{195} \approx 0.8\%.
\end{equation}

\subsection{Geometric derivation}

The quark-sector CP phase (the CKM angle $\gamma$) was derived in the
baryon-sector supplements as
\begin{equation}
  \gamma \;=\; \arctan\!\left(\frac{2\pi^2}{9}\right)
  \;\approx\; 65.5^{\circ}.
\end{equation}
This is the CP-violating phase acquired by a twisted-sector fermion
traversing a \emph{single} fold wall of the $\mathbb{Z}_3$ orbifold.

The leptonic CP phase differs by a factor of $p = 3$ (the order of the
$\mathbb{Z}_3$ group):

\begin{enumerate}[(i)]
  \item \textbf{Quarks (twisted sector):}  Quarks are pinned to the cone
    point and interact with only \emph{one} fold wall at a time.  The
    CP phase is $\gamma$ (single fold-wall contribution).

  \item \textbf{Neutrinos (untwisted sector):}  Neutrinos are neutral
    under $\mathbb{Z}_3$ and propagate freely through the bulk.  They
    access \emph{all three} fold walls simultaneously.  The CP phase is
    the coherent sum of three single-wall contributions:
    \begin{equation}
      \delta_{\mathrm{CP}}^{\mathrm{PMNS}}
        = 3\,\gamma = p\,\gamma.
    \end{equation}
\end{enumerate}

\begin{remark}
The factor of $3$ is exact (it is the group order $p = |\mathbb{Z}_3|$,
not an approximation).  The leptonic CP phase is predicted to be close to
but not exactly $180^{\circ}$ (which would correspond to maximal CP
violation in the PMNS matrix).  The deviation
$196.5^{\circ} - 180^{\circ} = 16.5^{\circ}$ encodes the non-trivial
value of $\arctan(2\pi^2/9)$.
\end{remark}

%% ====================================================================
\section{The Point/Side/Face Forward Derivation}
\label{sec:PSF}
%% ====================================================================

The central insight of the neutrino sector is that the three neutrino mass
eigenstates are \emph{not} three copies of one geometric object (as the
charged leptons are three copies of the cone-point mode with different
$\mathbb{Z}_3$ charges).  Instead, they are three \emph{different} geometric
objects within the $\mathbb{Z}_3$ orbifold:

\subsection{The three geometric objects}

\begin{definition}[Point/Side/Face assignment]
\label{def:PSF}
\mbox{}
\begin{itemize}
  \item $\nu_1$ = \textbf{Point} (cone tip, $0$-dimensional).
    Minimal geometric support $\Rightarrow$ lightest mass ($m_1 \approx 0$).

  \item $\nu_2$ = \textbf{Side} (fold wall, codimension-1 surface).
    Moderate tunnelling overlap $\Rightarrow$ intermediate mass.

  \item $\nu_3$ = \textbf{Face} (sector bulk, full-dimensional).
    Maximum tunnelling overlap $\Rightarrow$ heaviest mass.
\end{itemize}
\end{definition}

This assignment is not arbitrary: it is forced by the geometric hierarchy
of the orbifold.  The orbifold has exactly one cone point ($0$-dimensional),
three fold walls (codimension-1), and three sector bulks (full-dimensional).
The mass hierarchy $m_1 \ll m_2 < m_3$ reflects the hierarchy of geometric
support.

\subsection{Tunnelling overlap matrix}

The tunnelling overlap matrix $T$ encodes the amplitudes for neutrino
transitions between the three geometric objects.  Its entries are determined
by the orbifold invariants:

\begin{equation}
\label{eq:Tmatrix}
  T \;=\;
  \begin{pmatrix}
    T_{11} & T_{12} & T_{13} \\
    T_{12} & T_{22} & T_{23} \\
    T_{13} & T_{23} & T_{33}
  \end{pmatrix},
\end{equation}
where each entry is derived from the geometric invariants $\eta = 2/9$,
$K = 2/3$, $\sigma = d_1/(d_1+\lambda_1) = 6/11$, $p = 3$, and the
reactor angle $(\eta K)^2 = 16/729$.

\bigskip
\noindent\textbf{Diagonal entries:}

\begin{enumerate}[(i)]
  \item $T_{11} = +\dfrac{2\sigma}{p} = +\dfrac{2}{3}\cdot\dfrac{6}{11}
    = +\dfrac{4}{11}$
    \quad (point self-overlap: the cone tip sees both twisted sectors,
    giving a factor of $2$, divided by the orbifold order $p$).

  \item $T_{22} = -\dfrac{\sigma}{p} = -\dfrac{1}{3}\cdot\dfrac{6}{11}
    = -\dfrac{2}{11}$
    \quad (side self-overlap: the fold wall depletes its source state;
    a single twisted-sector contribution divided by $p$; sign negative
    because the tunnelling \emph{removes} amplitude from the source).

  \item $T_{33} = -(d_1 - 2)\cdot(\eta K)^2
    = -4 \cdot \dfrac{16}{729} = -\dfrac{64}{729}$
    \quad (face self-overlap: the bulk sector leaks through $d_1 - 2 = 4$
    transverse channels, each carrying the reactor-angle amplitude).
\end{enumerate}

\bigskip
\noindent\textbf{Off-diagonal entries:}

\begin{enumerate}[(i)]
  \item $T_{12} = -\eta = -\dfrac{2}{9}$
    \quad (point $\leftrightarrow$ side: the Donnelly invariant governs
    the transition between the cone tip and the adjacent fold wall).

  \item $T_{23} = -K\,\sigma = -\dfrac{2}{3}\cdot\dfrac{6}{11}
    = -\dfrac{4}{11}$
    \quad (side $\leftrightarrow$ face: the harmonic lock $K$ modulates the
    impedance ratio $\sigma$).

  \item $T_{13} = +\eta\, K = +\dfrac{4}{27}$
    \quad (point $\leftrightarrow$ face: requires two hops through
    the intervening side; the sign is positive because two negative
    transitions compose to a positive amplitude).
\end{enumerate}

\subsection{Mass matrix construction}

The perturbation strength is
\begin{equation}
  \varepsilon \;=\; \sqrt{d_1 + \lambda_1} \;=\; \sqrt{11}.
\end{equation}
The full neutrino mass matrix (in flavour basis) is
\begin{equation}
\label{eq:Mnu}
  M_\nu \;=\; D \;+\; \sqrt{11}\; T,
\end{equation}
where $D$ is the $3\times 3$ democratic matrix (all entries $1/3$).
The democratic matrix provides the zeroth-order (TBM) structure, and
$\sqrt{11}\, T$ is the geometric correction.

\subsection{Diagonalisation and PMNS extraction}

Numerical diagonalisation of $M_\nu$~\eqref{eq:Mnu} yields the PMNS mixing
matrix.  The predicted mixing parameters are compared with experiment in
Table~\ref{tab:PSF-results}.

\begin{table}[ht]
\centering
\begin{tabular}{@{} l c c c @{}}
\toprule
\textbf{Observable} & \textbf{Predicted} & \textbf{PDG/NuFIT}
  & \textbf{Deviation} \\
\midrule
$\sin^2\theta_{13}$ & $0.0216$ & $0.0220 \pm 0.0007$ & $-1.8\%$ \\
$\sin^2\theta_{12}$ & $0.303$  & $0.307 \pm 0.013$   & $-1.3\%$ \\
$\sin^2\theta_{23}$ & $0.537$  & $0.546 \pm 0.021$   & $-1.6\%$ \\
\bottomrule
\end{tabular}
\caption{PMNS mixing angles from the point/side/face forward model
compared with NuFIT~5.3 data~\cite{nufit2024}.  All predictions
are within $2\%$ of the central measured values.}
\label{tab:PSF-results}
\end{table}

\begin{remark}
The point/side/face model uses no free parameters beyond the orbifold
invariants ($\eta$, $K$, $d_1$, $\lambda_1$, $p$) already determined in
previous supplements.  The small deviations ($\lesssim 2\%$) from the exact
per-parameter formulae (Sections~\ref{sec:P21}--\ref{sec:P23}) reflect
the difference between isolated perturbation theory (exact formulae) and
the full matrix diagonalisation (which includes cross-couplings).
\end{remark}

%% ====================================================================
\section{Parameter 25 --- Neutrino Mass from the Inversion Principle}
\label{sec:P25}
%% ====================================================================

\subsection{Bulk resonance versus boundary tunnelling}

The proton and the neutrino represent two complementary aspects of the
orbifold geometry:

\begin{itemize}
  \item \textbf{Proton} = bulk resonance (constructive interference,
    interior standing wave).  The proton-to-electron mass ratio is
    \begin{equation}
      \frac{m_p}{m_e} = 6\pi^5
    \end{equation}
    (Supplement~IV).

  \item \textbf{Neutrino} = boundary tunnelling mode (evanescent wave,
    exponentially suppressed in the bulk).  The neutrino loses mass as
    the \emph{inverse} of the squared bulk volume, shared among $p$
    sectors.
\end{itemize}

\subsection{The master formula}

\begin{theorem}[Neutrino mass inversion]
\label{thm:numass}
\begin{equation}
\label{eq:inversion}
  \boxed{\;
  p\; m_p^2\; m_\nu \;=\; m_e^3.
  \;}
\end{equation}
\end{theorem}

\noindent Solving for the heaviest neutrino mass $m_3$:
\begin{equation}
\label{eq:m3}
  m_3 \;=\; \frac{m_e^3}{p\, m_p^2}
  \;=\; \frac{m_e}{p\,(m_p/m_e)^2}
  \;=\; \frac{m_e}{3\,(6\pi^5)^2}
  \;=\; \frac{m_e}{108\,\pi^{10}},
\end{equation}
where $108\,\pi^{10} = 3 \times (6\pi^5)^2$.

\subsection{Numerical verification}

\begin{align}
  6\pi^5 &= 1836.12, \\
  (6\pi^5)^2 &= 3{,}371{,}340, \\
  108\,\pi^{10} &= 3 \times 3{,}371{,}340 = 10{,}114{,}021, \\
  m_3\big|_{\mathrm{pred}}
    &= \frac{0.51100\;\mathrm{MeV}}{10{,}114{,}021}
    = 5.052 \times 10^{-8}\;\mathrm{MeV}
    = 50.52\;\mathrm{meV}.
\end{align}

The measured value (from oscillation data, normal ordering):
\begin{equation}
  m_3\big|_{\mathrm{meas}}
    = \sqrt{\Delta m^2_{32} + \Delta m^2_{21}}
    \approx \sqrt{2.453 \times 10^{-3} + 7.53 \times 10^{-5}}\;\mathrm{eV}
    = 50.28\;\mathrm{meV}.
\end{equation}

\begin{equation}
  \text{Deviation:}\quad
  \frac{|50.52 - 50.28|}{50.28} = 0.48\%.
\end{equation}

\subsection{Connection to the seesaw mechanism}

The inversion formula~\eqref{eq:inversion} can be rewritten in a form
reminiscent of the Type-I seesaw:
\begin{equation}
  m_\nu \;=\; \frac{m_e^3}{p\, m_p^2}
  \;=\; \frac{y_e^2\, v^2}{M_R},
\end{equation}
where the effective right-handed scale is
\begin{equation}
  M_R \;=\; \frac{p\, m_p^2}{m_e}
  \;\approx\; \frac{3 \times (0.938\;\mathrm{GeV})^2}{0.511 \times 10^{-3}\;\mathrm{GeV}}
  \;\approx\; 5.2 \times 10^{6}\;\mathrm{GeV}
  \;\approx\; 5.2 \times 10^{15}\;\mathrm{eV}.
\end{equation}

\begin{remark}
The scale $M_R \sim 5 \times 10^{15}$~eV is geometric in origin: it
represents the fold-wall penetration depth in the orbifold geometry.
It is \emph{not} the mass of a physical right-handed neutrino particle.
The seesaw form $m_\nu = y^2 v^2/M_R$ is recovered as a mathematical
identity, but the underlying physics is tunnelling suppression rather
than heavy-particle exchange.
\end{remark}

%% ====================================================================
\section{Parameter 26 --- Mass-Squared Splitting Ratio}
\label{sec:P26}
%% ====================================================================

\subsection{The formula}

\begin{theorem}[Mass-squared ratio]
\label{thm:ratio}
\begin{equation}
\label{eq:ratio}
  \boxed{\;
  \frac{\Delta m^2_{32}}{\Delta m^2_{21}}
    \;=\; d_1^2 - p
    \;=\; 36 - 3
    \;=\; 33.
  \;}
\end{equation}
\end{theorem}

\subsection{Numerical comparison}

\begin{equation}
  \frac{\Delta m^2_{32}}{\Delta m^2_{21}}\bigg|_{\mathrm{pred}} = 33.
\end{equation}
The measured value (NuFIT~5.3~\cite{nufit2024}):
\begin{equation}
  \frac{\Delta m^2_{32}}{\Delta m^2_{21}}\bigg|_{\mathrm{meas}}
    = \frac{2.453 \times 10^{-3}\;\mathrm{eV}^2}
           {7.53 \times 10^{-5}\;\mathrm{eV}^2}
    = 32.58 \pm 0.80.
\end{equation}
\begin{equation}
  \text{Deviation:}\quad
  \frac{|33 - 32.58|}{32.58} = 1.3\%.
\end{equation}

\subsection{Geometric derivation}

The ratio of mass-squared splittings reflects the ratio of tunnelling
bandwidths for the atmospheric and solar channels:

\begin{enumerate}[(i)]
  \item \textbf{Atmospheric splitting ($\Delta m^2_{32}$).}
    The 2--3 transition involves the full tunnelling bandwidth of the
    $\ell = 1$ sector.  The number of available two-body tunnelling
    channels scales as $d_1^2 = 6^2 = 36$, counting all pairwise
    mode combinations.

  \item \textbf{Solar splitting ($\Delta m^2_{21}$).}
    The 1--2 transition is a subtler, single fold-wall process.  Its
    bandwidth is the baseline against which the atmospheric bandwidth
    is measured.

  \item \textbf{Three-fold sharing.}
    The ratio is reduced by $p = 3$ because the three sectors of the
    $\mathbb{Z}_3$ orbifold share the total tunnelling bandwidth equally:
    \begin{equation}
      \frac{\Delta m^2_{32}}{\Delta m^2_{21}}
        = d_1^2 - p = 36 - 3 = 33.
    \end{equation}
\end{enumerate}

\subsection{Derived masses from P25 + P26}

Combining the heaviest neutrino mass (Parameter~25) with the splitting
ratio (Parameter~26) determines the full mass spectrum:

\begin{align}
  m_3 &= 50.52\;\mathrm{meV}, \\
  \frac{m_3^2 - m_2^2}{m_2^2 - m_1^2} &= 33.
\end{align}

In the normal-ordering limit $m_1 \approx 0$:
\begin{align}
  m_3^2 - m_2^2 &= 33\, m_2^2, \\
  m_3^2 &= 34\, m_2^2, \\
  m_2 &= \frac{m_3}{\sqrt{34}} = \frac{50.52}{\sqrt{34}}
    = \frac{50.52}{5.831} = 8.66\;\mathrm{meV}.
\end{align}

The measured value:
\begin{equation}
  m_2\big|_{\mathrm{meas}} = \sqrt{\Delta m^2_{21}}
    = \sqrt{7.53 \times 10^{-5}}\;\mathrm{eV} = 8.68\;\mathrm{meV}.
\end{equation}
\begin{equation}
  \text{Deviation:}\quad
  \frac{|8.66 - 8.68|}{8.68} = 0.23\%.
\end{equation}

The mass spectrum is:
\begin{equation}
  m_1 \approx 0, \qquad
  m_2 = 8.66\;\mathrm{meV}, \qquad
  m_3 = 50.52\;\mathrm{meV}.
\end{equation}
This is the \textbf{normal hierarchy}, predicted by the point/side/face
geometric assignment (the point mode has minimal support and therefore
minimal mass).

The sum of neutrino masses:
\begin{equation}
  \sum m_\nu \;\approx\; 0 + 8.66 + 50.52 \;=\; 59.2\;\mathrm{meV}.
\end{equation}

%% ====================================================================
\section{Why No Neutrino Koide Ratio}
\label{sec:noKoide}
%% ====================================================================

The charged-lepton Koide ratio $K = 2/3$ is exact.  One might ask whether
an analogous relation holds for neutrinos.  It does not, and the reason is
structural.

\subsection{Charged leptons: circulant symmetry}

The three charged-lepton masses are eigenvalues of a $3 \times 3$ Hermitian
circulant matrix (Supplement~II, \S1.1).  The circulant structure arises
because all three generations are the \emph{same geometric object} (the
cone-point mode) carrying different $\mathbb{Z}_3$ charges
$\omega^0, \omega^1, \omega^2$.  The Koide identity
\begin{equation}
  K = \frac{\sum m_k}{\bigl(\sum\sqrt{m_k}\bigr)^2} = \frac{2}{3}
\end{equation}
is a \emph{trace identity} for circulant matrices with eigenvalue modulus
$r = \sqrt{2}$, valid for any value of the phase $\delta$.

\subsection{Neutrinos: three different objects}

The three neutrino mass eigenstates are three \emph{different} geometric
objects (Definition~\ref{def:PSF}):
\begin{itemize}
  \item $\nu_1$ = point (0-dimensional),
  \item $\nu_2$ = side (codimension-1),
  \item $\nu_3$ = face (full-dimensional).
\end{itemize}
Their mass matrix~\eqref{eq:Mnu} is \emph{not} circulant: $T_{11} \neq
T_{22} \neq T_{33}$ and the off-diagonal entries are not related by cyclic
permutation.  Therefore the circulant trace identity does not apply.

\subsection{The neutrino $Q_\nu$}

Computing the Koide-like ratio for neutrinos (using $m_1 \approx 0$,
$m_2 = 8.66$~meV, $m_3 = 50.52$~meV):
\begin{equation}
  Q_\nu \;=\; \frac{m_1 + m_2 + m_3}
    {(\sqrt{m_1} + \sqrt{m_2} + \sqrt{m_3})^2}
  \;\approx\; \frac{59.2}{(0 + 2.943 + 7.108)^2}
  \;=\; \frac{59.2}{101.0}
  \;\approx\; 0.586.
\end{equation}

\begin{proposition}[$Q_\nu \neq 2/3$ is a prediction]
\label{prop:Qnu}
The inequality $Q_\nu \neq 2/3$ is not a failure of the framework; it is
a \emph{prediction}.  The value $Q_\nu \approx 0.586$ follows from the
point/side/face mass hierarchy.  Measurement of a neutrino Koide-like ratio
close to $2/3$ would \emph{falsify} the geometric model.
\end{proposition}

%% ====================================================================
\section{Derived Predictions}
\label{sec:predictions}
%% ====================================================================

The six parameters derived in this supplement, combined with the mass
spectrum, yield two cosmologically testable predictions.

\subsection{Sum of neutrino masses}

\begin{equation}
  \boxed{\;
  \sum m_\nu \;=\; m_1 + m_2 + m_3
    \;\approx\; 0 + 8.66 + 50.52
    \;=\; 59.2\;\mathrm{meV}.
  \;}
\end{equation}

This lies squarely within the sensitivity window of forthcoming
cosmological surveys.  The DESI baryon acoustic oscillation programme and
the Euclid satellite are projected to constrain $\sum m_\nu$ to
$\sim 50$--$70$~meV at $95\%$~CL~\cite{pdg2024}.  A measured value
significantly below $50$~meV (inverted hierarchy) or significantly above
$70$~meV would be in tension with the prediction.

\subsection{Effective Majorana mass}

If neutrinos are Majorana particles, the effective mass governing
neutrinoless double-beta decay ($0\nu\beta\beta$) is
\begin{equation}
  |m_{\beta\beta}| \;=\; \left|\sum_k U_{ek}^2\, m_k\right|.
\end{equation}
In the normal hierarchy with $m_1 \approx 0$:
\begin{equation}
  \boxed{\;
  |m_{\beta\beta}| \;\approx\; 2\text{--}3\;\mathrm{meV}.
  \;}
\end{equation}

This is below the sensitivity of current experiments (KamLAND-Zen,
$|m_{\beta\beta}| < 36$--$156$~meV) but within the projected reach of
next-generation experiments:
\begin{itemize}
  \item nEXO: sensitivity $\sim 5$--$17$~meV,
  \item LEGEND-1000: sensitivity $\sim 9$--$21$~meV.
\end{itemize}

A positive signal in the $2$--$3$~meV range would require further
detector improvements beyond the next generation.  However, a signal above
$\sim 10$~meV in normal ordering would be in tension with this framework.

%% ====================================================================
\section{Provenance Table}
\label{sec:provenance}
%% ====================================================================

Table~\ref{tab:provenance} maps every result in this supplement to its
mathematical source, verification method, and epistemic status.

\begin{table}[ht]
\centering
\small
\begin{tabular}{@{} p{3.8cm} p{3.2cm} p{3.0cm} c @{}}
\toprule
\textbf{Result} & \textbf{Mathematical Source}
  & \textbf{Verification} & \textbf{Status} \\
\midrule
Twisted/untwisted sectors (Table~\ref{tab:twisted-untwisted})
  & $\mathbb{Z}_3$ orbifold topology
  & Structural identification
  & Framework \\[4pt]
TBM from $\mathbb{Z}_3$ (Prop.~\ref{prop:TBM})
  & Eigenvectors of cyclic permutation
  & Direct linear algebra
  & Theorem \\[4pt]
$\sin^2\theta_{13} = 16/729$ (Thm.~\ref{thm:reactor})
  & $(\eta K)^2$; junction asymmetry
  & $0.02194$ vs $0.02200$ ($0.27\%$)
  & Prediction \\[4pt]
$\sin^2\theta_{12} = 25/81$ (Thm.~\ref{thm:solar})
  & $1/3 - \eta^2/2$; fold-wall thickness
  & $0.3086$ vs $0.307$ ($0.53\%$)
  & Prediction \\[4pt]
$\sin^2\theta_{23} = 6/11$ (Thm.~\ref{thm:atmospheric})
  & $d_1/(d_1+\lambda_1)$; impedance ratio
  & $0.5455$ vs $0.546$ ($0.10\%$)
  & Prediction \\[4pt]
$\delta_{\mathrm{CP}} = 196.5^{\circ}$ (Thm.~\ref{thm:deltaCP})
  & $3\arctan(2\pi^2/9)$; $p \times$ quark phase
  & $196.5^{\circ}$ vs $195^{\circ}$ ($0.8\%$)
  & Prediction \\[4pt]
Point/side/face model (Def.~\ref{def:PSF})
  & Orbifold geometric hierarchy
  & Table~\ref{tab:PSF-results} (all $<2\%$)
  & Derived \\[4pt]
Tunnelling matrix $T$ (Eq.~\ref{eq:Tmatrix})
  & Orbifold invariants $\eta, K, \sigma, p$
  & PMNS extraction
  & Derived \\[4pt]
$m_3 = m_e/(108\pi^{10})$ (Thm.~\ref{thm:numass})
  & Inversion: $p\,m_p^2\,m_\nu = m_e^3$
  & $50.52$ vs $50.28$~meV ($0.48\%$)
  & Prediction \\[4pt]
$\Delta m^2_{32}/\Delta m^2_{21} = 33$ (Thm.~\ref{thm:ratio})
  & $d_1^2 - p$; tunnelling bandwidth
  & $33$ vs $32.58$ ($1.3\%$)
  & Prediction \\[4pt]
$m_2 = m_3/\sqrt{34}$ (derived)
  & P25 + P26 combination
  & $8.66$ vs $8.68$~meV ($0.23\%$)
  & Derived \\[4pt]
$Q_\nu \neq 2/3$ (Prop.~\ref{prop:Qnu})
  & Non-circulant mass matrix
  & $Q_\nu \approx 0.586$
  & Prediction \\[4pt]
$\sum m_\nu \approx 59.2$~meV
  & Mass spectrum sum
  & DESI/Euclid window
  & Prediction \\[4pt]
$|m_{\beta\beta}| \sim 2$--$3$~meV
  & PMNS elements + masses
  & nEXO/LEGEND-1000 reach
  & Prediction \\[4pt]
Seesaw form recovered
  & $M_R = p\,m_p^2/m_e$
  & $M_R \sim 5.2 \times 10^{15}$~eV
  & Consistency \\
\bottomrule
\end{tabular}
\caption{Provenance map for Supplement~VII results (Parameters~21--26 and
derived predictions).
``Theorem'' entries follow from established mathematics.
``Derived'' entries follow algebraically from prior results.
``Prediction'' entries are compared against PDG/NuFIT measurements.
``Framework'' entries depend on the orbifold identification.
``Consistency'' entries recover known structures.}
\label{tab:provenance}
\end{table}

\subsection{Connection to the lotus potential}

The neutrino sector parameters (P21--P26) are evaluated at the lotus point $\phi_{\mathrm{lotus}} = 0.9574$ of the fold potential $V(\phi) = \lambda_H v_{\max}^4 (\phi^2 - \phi_{\mathrm{lotus}}^2)^2/4$. The neutrino masses arise from the petal overlap: ghost wavefunctions bleed through the $4.3\%$ residual fold opening, with the lightest neutrino $m_1 \approx 0$ corresponding to the minimal tunneling amplitude at the cone tip (point mode). The cosmological constant $\Lambda^{1/4} = m_{\nu_3}\eta^2 = 2.49$~meV is the infinitesimal breathing energy of the lotus --- nonzero because $\phi < 1$ (the fold never fully closes).

%% ====================================================================
\begin{thebibliography}{99}

\bibitem{pdg2024}
R.~L.~Workman \textit{et al.}\ (Particle Data Group),
``Review of Particle Physics,''
\textit{Prog.\ Theor.\ Exp.\ Phys.}\ \textbf{2022} (2022) 083C01,
and 2024 update.

\bibitem{nufit2024}
I.~Esteban, M.~C.~Gonzalez-Garcia, M.~Maltoni, T.~Schwetz, and A.~Zhou,
``The fate of hints: updated global analysis of three-flavour neutrino
oscillations,''
\textit{J.\ High Energy Phys.}\ \textbf{09} (2020) 178;
NuFIT~5.3 (2024), \texttt{www.nu-fit.org}.

\bibitem{donnelly1978}
H.~Donnelly,
``Eta invariants for $G$-spaces,''
\textit{Indiana Univ.\ Math.\ J.}\ \textbf{27} (1978) 889--918.

\end{thebibliography}

\end{document}
