\documentclass[12pt]{article}
\usepackage{amsmath,amssymb,amsthm}
\usepackage{geometry}
\usepackage{booktabs}
\usepackage{parskip}

\geometry{margin=1.2in}

\newtheorem{theorem}{Theorem}
\newtheorem{corollary}{Corollary}
\newtheorem{proposition}{Proposition}
\newtheorem{definition}{Definition}
\newtheorem{remark}{Remark}

\title{\textbf{Supplement X: The Spectral Action Derivation}\\[0.3em]
\large The Dictionary, the Identity Chain, and the Cosmological Constant\\[0.2em]
\normalsize The Resolved Chord --- Supplementary Material}

\author{Jixiang Leng}
\date{February 2026}

\begin{document}
\maketitle

\noindent\textit{This supplement provides the complete derivation chains for the spectral dictionary
(Section~2 of the main text), the gravity identity chain (Section~11), the cosmological
constant (Section~12), and the force unification picture (Section~13).
All computations are verified by the scripts indexed in \texttt{MASTER\_CODE\_INDEX.md}.}

%% ============================================================
\section{The Spectral Dictionary Derivation}
%% ============================================================

\subsection{Level 1: The proton mass decomposition}

\begin{theorem}[$\pi^2 = \lambda_1 + \alpha_s$]\label{thm:pi2}
Let $\lambda_1 = \ell(\ell+4)|_{\ell=1} = 5$ be the first nonzero eigenvalue of the scalar
Laplacian on $S^5$ (Ikeda~1980).  Let $\alpha_s = \pi^2 - 5$ be the Dirichlet spectral gap.
Then $\pi^2 = \lambda_1 + \alpha_s$, where both summands have independent geometric meaning:
$\lambda_1$ is the kinetic energy per ghost mode; $\alpha_s$ is the strong coupling
(after RG running to $M_Z$: $\alpha_s(M_Z) = 0.1174$, $0.6\sigma$ from PDG).
\end{theorem}

\begin{proof}
The identity $\pi^2 = 5 + (\pi^2 - 5)$ is algebraic.  The content is:
(i)~$\lambda_1 = 5$ is a theorem of spectral geometry~(Ikeda);
(ii)~$\alpha_s = \pi^2 - 5$ is the Dirichlet gap identified with the strong coupling
(Parameter~9 of the main text).
\end{proof}

\begin{corollary}[Proton decomposition]
The tree-level proton mass is $m_p/m_e = d_1 \cdot \mathrm{Vol}(S^5) \cdot \pi^2 = 6\pi^5$,
where $d_1 = 6$ (ghost mode count), $\mathrm{Vol}(S^5) = \pi^3$, and $\pi^2 = \lambda_1 + \alpha_s$.
This equals the Gaussian phase-space integral over $\mathbb{R}^{10}$ (Supplement~IV, \S1):
both the local (Gaussian) and global (Vol $\times$ energy) pictures give $\pi^5$.
\end{corollary}

\noindent\textbf{Verification:} \texttt{spectral\_action\_dictionary.py}.

\subsection{Level 2: The fine-structure constant}

The lag correction $G/p = \lambda_1\eta/p = 10/27$ is derived in Supplement~IV: the spectral coupling $G = \lambda_1\eta = 10/9$ in \S5--6, the lag mechanism in \S8.2, and the non-circular inversion extracting $\alpha$ from the proton mass ratio in \S7.
Combined with $\sin^2\theta_W = 3/8$ (SO(6) branching) and SM two-loop running,
this gives $1/\alpha(0) = 137.038$ ($0.001\%$).

\subsection{Level 3: The Higgs sector}

\begin{proposition}[Dirac eigenvalue at ghost level]\label{prop:dirac}
On the round unit $S^5$, the Dirac eigenvalues are $\pm(\ell + 5/2)$.
At the ghost level $\ell = 1$: $\lambda_1^D = 7/2$.
\end{proposition}
\begin{proof}
Standard result (Ikeda~1980, Gilkey~1984): on $S^{2k+1}$, eigenvalues $\pm(\ell+k+1/2)$;
for $S^5$ ($k=2$): $\pm(\ell+5/2)$; at $\ell=1$: $\pm 7/2$.
\end{proof}

\noindent The Higgs formulas (Supplement~V):
$v/m_p = 2/\alpha - (d_1+\lambda_1+K) = 2/\alpha - 35/3$ (two twisted sectors, ghost cost);
$m_H/m_p = 1/\alpha - 7/2$ (one sector excitation, Dirac eigenvalue);
$\lambda_H = (m_H/m_p)^2/[2(v/m_p)^2] = 0.1295$.

%% ============================================================
\section{The Identity Chain}
%% ============================================================

\begin{theorem}[$\eta = d_1/p^n$]\label{thm:eta-ghost}
The Donnelly eta invariant on $S^5/\mathbb{Z}_3$ equals the ghost mode count per orbifold volume:
\[
\eta = \sum_{m=1}^{p-1}|\eta_D(\chi_m)| = \frac{d_1}{p^n} = \frac{6}{27} = \frac{2}{9}.
\]
\end{theorem}
\begin{proof}
Direct computation from the Donnelly formula (Supplement~I, \S2):
$|\eta_D(\chi_1)| = |\eta_D(\chi_2)| = 1/9$; sum $= 2/9$.
And $d_1/p^n = 6/27 = 2/9$.  The identity holds because $d_1 = 2n$ and
$p^n = 27$ for $(n,p)=(3,3)$, with $\eta = 2n/p^n = 2/9$.
\end{proof}

\noindent From this single identity:
\begin{align}
\tau &= 1/p^n = 1/27 &&\text{(Reidemeister torsion)}, \\
G &= \lambda_1 \eta = 10/9 &&\text{(proton coupling)}, \\
c_{\mathrm{grav}} &= -\tau/G = -1/(d_1\lambda_1) = -1/30 &&\text{(gravity = topology $\div$ QCD)}.
\end{align}

\noindent\textbf{Verification:} \texttt{gravity\_derivation\_v3.py}.

%% ============================================================
\section{The Cosmological Constant Derivation}
%% ============================================================

\begin{theorem}[CC from round-trip tunneling]\label{thm:cc}
The one-loop cosmological constant on $(B^6/\mathbb{Z}_3, S^5/\mathbb{Z}_3)$ is:
\[
\Lambda^{1/4} = m_{\nu_3} \cdot \eta^2 \cdot \left(1 - \frac{K}{d_1}\right)
= m_{\nu_3} \cdot \frac{32}{729} = 2.22\;\mathrm{meV} \quad (1.4\%).
\]
\end{theorem}

\noindent\textbf{Derivation:}
\begin{enumerate}
\item[\textbf{(i)}] $V_{\mathrm{tree}}(\phi_{\mathrm{lotus}}) = 0$ (orbifold volume cancellation). \textit{[Theorem.]}
\item[\textbf{(ii)}] One-loop CC from twisted sectors only (renormalization absorbs untwisted). \textit{[Derived.]}
\item[\textbf{(iii)}] Heavy mode cancellation: $2\mathrm{Re}[\chi_l(\omega)] \to 0$ for $l \gg 1$
(equidistribution of $\mathbb{Z}_3$ characters; verified to $l = 500$). \textit{[Verified.]}
\item[\textbf{(iv)}] Neutrino dominance: $m_{\nu_3} = m_e/(108\pi^{10})$ is the lightest tunneling mode. \textit{[Derived.]}
\item[\textbf{(v)}] Round-trip tunneling: the one-loop bubble crosses the boundary twice;
APS boundary condition gives amplitude $\eta$ per crossing; round trip $= \eta^2 = 4/81$.
Consistency: odd Dedekind sums vanish for $\mathbb{Z}_3$ ($\cot^3(\pi/3) + \cot^3(2\pi/3) = 0$),
confirming even (squared) order. \textit{[Derived.]}
\item[\textbf{(vi)}] Koide absorption: $K/d_1 = (2/p)/(2p) = 1/p^2 = 1/9$;
residual $(1 - 1/p^2) = 8/9$. \textit{[Theorem.]}
\item[\textbf{(vii)}] Result: $\Lambda^{1/4} = 50.52\;\mathrm{meV} \times 32/729 = 2.22\;\mathrm{meV}$.
Observed: $2.25\;\mathrm{meV}$ ($1.4\%$). \textit{[Derivation.]}
\end{enumerate}

\noindent\textbf{Why the CC is small:}
(a)~Heavy modes cancel (equidistribution).
(b)~Only $m_{\nu_3}$ survives ($50$~meV, not $100$~GeV).
(c)~Double boundary crossing: $\eta^2 = 4/81$.
(d)~Koide absorption: $8/9$.
Combined: $50 \times 0.044 = 2.2$~meV.  Not fine-tuning --- geometry.

\noindent\textbf{Verification:} \texttt{cc\_aps\_proof.py}, \texttt{cc\_monogamy\_cancellation.py}.

%% ============================================================
\section{Provenance Table}
%% ============================================================

\begin{table}[h]
\centering
\small
\begin{tabular}{p{4.5cm}p{3cm}p{2.5cm}l}
\toprule
\textbf{Result} & \textbf{Source} & \textbf{Verification} & \textbf{Status} \\
\midrule
$\pi^2 = \lambda_1 + \alpha_s$ & Algebraic + Ikeda & Exact & Theorem \\
$\eta = d_1/p^n = 2/9$ & Donnelly + counting & $<10^{-10}$ & Theorem \\
$c_{\mathrm{grav}} = -\tau/G = -1/30$ & Identity chain & $M_P$ to $0.37\%$ & Derived \\
$X_{\mathrm{bare}} = (d_1+\lambda_1)^2/p$ & Heat kernel $a_2$ & $3.4\%$ bare & Identified \\
$7/2$ = Dirac at ghost level & Ikeda 1980 & Algebraic & Theorem \\
$\Lambda^{1/4} = m_{\nu_3} \cdot 32/729$ & Round-trip tunneling & $1.4\%$ & Derived \\
Heavy mode cancellation & Equidistribution & $l = 0\ldots500$ & Verified \\
$K/d_1 = 1/p^2 = 1/9$ & Algebra: $K=2/p, d_1=2p$ & Exact & Theorem \\
\bottomrule
\end{tabular}
\caption{Provenance map for Supplement~X results.}
\end{table}

\begin{thebibliography}{99}

\bibitem{donnelly1978}
H.~Donnelly, ``Eta invariants for $G$-spaces,''
\textit{Indiana Univ.\ Math.\ J.}\ \textbf{27} (1978) 889--918.

\bibitem{ikeda1980}
A.~Ikeda, ``On the spectrum of the Laplacian on the spherical space forms,''
\textit{Osaka J.\ Math.}\ \textbf{17} (1980) 691.

\bibitem{gilkey1984}
P.~B.~Gilkey, \textit{Invariance Theory, the Heat Equation, and the Atiyah-Singer Index Theorem},
Publish or Perish, 1984.

\bibitem{grubb1996}
G.~Grubb, \textit{Functional Calculus of Pseudodifferential Boundary Problems},
2nd ed., Birkh\"auser, 1996.

\bibitem{cheeger1979}
J.~Cheeger, ``Analytic torsion and the heat equation,''
\textit{Ann.\ of Math.}\ \textbf{109} (1979) 259--322.

\end{thebibliography}

\end{document}
