\documentclass[12pt]{article}
\usepackage{amsmath,amssymb,amsthm}
\usepackage{geometry}
\usepackage{booktabs}
\usepackage{hyperref}

\geometry{margin=1.2in}

\hypersetup{colorlinks=true, linkcolor=blue, citecolor=blue}

\newtheorem{theorem}{Theorem}[section]
\newtheorem{lemma}[theorem]{Lemma}
\newtheorem{corollary}[theorem]{Corollary}
\newtheorem{proposition}[theorem]{Proposition}
\newtheorem{definition}[theorem]{Definition}
\newtheorem{remark}[theorem]{Remark}

\title{\textbf{Spectral Exclusion on Orbifolded Spheres\\and the Absence of Fundamental Triplets}}

\author{Jixiang Leng}
\date{February 2026}

\begin{document}
\maketitle

\begin{abstract}
We prove that on the orbifold $S^5/\mathbb{Z}_3$, the $\ell = 1$ eigenspace
of the scalar Laplacian contains no $\mathbb{Z}_3$-invariant modes.
All six first-level harmonics transform non-trivially under $\mathbb{Z}_3$:
three in the character $\chi_1$ and three in $\chi_2$.
This ``triple spectral exclusion'' has three consequences:
(i)~the physical spectrum has a mass gap at $\ell = 1$ (the ghost modes are absent);
(ii)~no massless fundamental $\mathbf{3}$ of $\mathrm{SU}(3)$ propagates on the orbifold
(confinement);
(iii)~chirality is enforced because the ghost modes break the left-right spectral symmetry
through the eta invariant $\eta = 2/9 \neq 0$.
The proof uses only the representation theory of $\mathbb{Z}_3$ acting on
$\mathbb{C}^3$ and the harmonic analysis of $S^5 \subset \mathbb{C}^3$.
\end{abstract}

%% ============================================================
\section{The Harmonics of \texorpdfstring{$S^5$}{S5}}
%% ============================================================

\begin{definition}[Spherical harmonics on $S^{2n-1}$]\label{def:harmonics}
The eigenspaces of the scalar Laplacian $-\Delta$ on the round unit $S^{2n-1}$
at level $\ell$ are the restrictions to $S^{2n-1}$ of harmonic homogeneous polynomials
of degree $\ell$ on $\mathbb{R}^{2n} = \mathbb{C}^n$.  The eigenvalue is
$\lambda_\ell = \ell(\ell + 2n - 2)$ and the degeneracy is
$d_\ell = \binom{\ell+2n-1}{\ell} - \binom{\ell+2n-3}{\ell-2}$.
\end{definition}

For $S^5$ ($n = 3$): $\lambda_0 = 0$, $d_0 = 1$; $\lambda_1 = 5$, $d_1 = 6$;
$\lambda_2 = 12$, $d_2 = 20$.

\begin{proposition}[$\ell = 1$ harmonics are linear functions]\label{prop:ell1}
The $\ell = 1$ eigenspace of $-\Delta$ on $S^5 \subset \mathbb{C}^3$ is spanned by
the restrictions of the six real-linear coordinate functions:
\begin{equation}
\{x_1, y_1, x_2, y_2, x_3, y_3\} \quad\text{where } z_j = x_j + iy_j.
\end{equation}
Equivalently, as complex-linear and anti-linear functions:
$\{z_1, z_2, z_3, \bar{z}_1, \bar{z}_2, \bar{z}_3\}$.
\end{proposition}

\begin{proof}
Standard: the harmonic homogeneous polynomials of degree 1 on $\mathbb{R}^6$ are
the linear functions, forming a 6-dimensional space.
On $S^5$: $-\Delta(z_j|_{S^5}) = \lambda_1 \cdot z_j|_{S^5}$ with $\lambda_1 = 1 \cdot (1+4) = 5$.  \qed
\end{proof}

%% ============================================================
\section{The \texorpdfstring{$\mathbb{Z}_3$}{Z3} Action on the \texorpdfstring{$\ell = 1$}{l=1} Eigenspace}
%% ============================================================

\begin{definition}[$\mathbb{Z}_3$ action]\label{def:z3-action}
$\mathbb{Z}_3$ acts on $\mathbb{C}^3$ by $\omega \cdot (z_1, z_2, z_3) = (\omega z_1, \omega z_2, \omega z_3)$
with $\omega = e^{2\pi i/3}$.  This induces an action on the $\ell = 1$ eigenspace:
\begin{align}
\omega \cdot z_j &= \omega z_j \qquad\text{(character $\chi_1$: eigenvalue $\omega$)}, \\
\omega \cdot \bar{z}_j &= \bar{\omega}\, \bar{z}_j = \omega^2 \bar{z}_j
\qquad\text{(character $\chi_2$: eigenvalue $\omega^2$)}.
\end{align}
\end{definition}

\begin{theorem}[Triple Spectral Exclusion]\label{thm:exclusion}
The $\ell = 1$ eigenspace of $-\Delta$ on $S^5$ contains \textbf{no} $\mathbb{Z}_3$-invariant
modes.  The six-dimensional space decomposes as:
\begin{equation}
\boxed{H_1 = V_{\chi_1} \oplus V_{\chi_2}, \qquad \dim V_{\chi_1} = 3, \qquad \dim V_{\chi_2} = 3, \qquad \dim V_{\chi_0} = 0.}
\end{equation}
All $d_1 = 6$ modes are \textbf{ghost modes} (non-invariant under $\mathbb{Z}_3$).
\end{theorem}

\begin{proof}
The six basis elements decompose under $\mathbb{Z}_3$ as:
\begin{center}
\begin{tabular}{lccc}
\toprule
\textbf{Basis element} & \textbf{$\omega$ eigenvalue} & \textbf{Character} & \textbf{Invariant?} \\
\midrule
$z_1, z_2, z_3$ & $\omega$ & $\chi_1$ & No \\
$\bar{z}_1, \bar{z}_2, \bar{z}_3$ & $\omega^2$ & $\chi_2$ & No \\
\bottomrule
\end{tabular}
\end{center}
No linear combination of $\chi_1$-modes and $\chi_2$-modes can be $\chi_0$-invariant
(since $\chi_1 \neq \chi_0$ and $\chi_2 \neq \chi_0$, and the characters are orthogonal
in $\mathbb{C}[\mathbb{Z}_3]$).  Therefore $V_{\chi_0} = \{0\}$: no invariant modes.  \qed
\end{proof}

\begin{remark}[Why the diagonal action is special]
If $\mathbb{Z}_3$ acted with different weights, e.g., $(z_1, z_2, z_3) \mapsto (\omega z_1, \omega z_2, z_3)$,
then $z_3$ and $\bar{z}_3$ would be invariant, and $V_{\chi_0}$ would be 2-dimensional.
The \emph{diagonal} action (all weights equal to 1) is the one that kills ALL $\ell = 1$ modes.
This is the action selected by the uniqueness theorem $n = p^{n-2}$~\cite{leng2026eta}.
\end{remark}

%% ============================================================
\section{Consequences}
%% ============================================================

\subsection{The mass gap}

\begin{corollary}[Ghost spectral gap]\label{cor:gap}
On $S^5/\mathbb{Z}_3$, the physical (invariant) scalar spectrum has a gap:
the first nonzero invariant eigenvalue is at $\ell = 2$ ($\lambda_2 = 12$),
not $\ell = 1$ ($\lambda_1 = 5$).  The $\ell = 1$ level is entirely removed
from the physical spectrum.
\end{corollary}

\begin{proof}
Theorem~\ref{thm:exclusion}: $d_1^{(0)} = 0$.  The first level with invariant modes
is $\ell = 2$, where $d_2^{(0)} = 8$ (computed by the character formula:
$d_2 = 20$, and $20/3 + \text{character corrections} = 8$).  \qed
\end{proof}

\subsection{Confinement}

\begin{corollary}[No fundamental triplet]\label{cor:confinement}
The three holomorphic coordinates $z_1, z_2, z_3$ transform in the
fundamental $\mathbf{3}$ of $\mathrm{SU}(3) \subset \mathrm{SO}(6) = \mathrm{Isom}(S^5)$,
where $\mathrm{SU}(3)$ is embedded via its natural action on $\mathbb{C}^3$.
The diagonal $\mathbb{Z}_3$ used throughout this paper is the \emph{center}
$Z(\mathrm{SU}(3)) \cong \mathbb{Z}_3$, acting as scalar multiplication on the $\mathbf{3}$.
Since they are all in $V_{\chi_1}$ (non-invariant), no physical mode at $\ell = 1$
transforms in the fundamental $\mathbf{3}$.  A free color triplet cannot propagate
on $S^5/\mathbb{Z}_3$: it is confined.
\end{corollary}

\begin{proof}
A physical (propagating) mode must be $\mathbb{Z}_3$-invariant.
The $\mathbf{3}$ of SU(3) lies entirely in $V_{\chi_1}$ at $\ell = 1$.
Therefore no $\mathbb{Z}_3$-invariant mode transforms as a fundamental triplet.
Color singlet combinations arise at higher $\ell$
(e.g., from $\bar{\mathbf{3}} \otimes \mathbf{3}$ decompositions at $\ell = 2$).
The key point is that no \emph{fundamental} $\mathbf{3}$ propagates at $\ell = 1$.  \qed
\end{proof}

\subsection{Chirality}

\begin{corollary}[Chirality from exclusion]\label{cor:chirality}
The splitting $H_1 = V_{\chi_1} \oplus V_{\chi_2}$ with $3 + 3$ (rather than $6 + 0$
or $2 + 2 + 2$) breaks left-right symmetry.  The $\chi_1$ and $\chi_2$ sectors
contribute with opposite signs to the eta invariant:
$\eta_D(\chi_1) = +1/9$ and $\eta_D(\chi_2) = -1/9$~\cite{leng2026eta}.
The nonvanishing total $\eta = 2/9 \neq 0$ is the spectral signature of chirality.
\end{corollary}

%% ============================================================
\section{Higher Levels}
%% ============================================================

\begin{proposition}[Character decomposition at $\ell = 2, 3$]\label{prop:higher}
At $\ell = 2$: $d_2 = 20$, $d_2^{(0)} = 8$, $d_2^{\mathrm{ghost}} = 12$.
At $\ell = 3$: $d_3 = 50$, $d_3^{(0)} = 20$, $d_3^{\mathrm{ghost}} = 30$.
\end{proposition}

\begin{proof}
The $\ell$-th harmonic space on $S^5$ is spanned by polynomials
$z_1^{a_1}z_2^{a_2}z_3^{a_3}\bar{z}_1^{b_1}\bar{z}_2^{b_2}\bar{z}_3^{b_3}$
with $\sum a_j + \sum b_j = \ell$ (restricted to the harmonic subspace).
Under $\mathbb{Z}_3$, such a monomial transforms with character
$\omega^{(\sum a_j - \sum b_j) \bmod 3}$.  The invariant monomials are those
with $\sum a_j \equiv \sum b_j \pmod{3}$.

For $\ell = 2$: the invariant count is $d_2^{(0)} = (d_2 + 2\mathrm{Re}[\chi_2(\omega)])/3$,
where $\chi_2(\omega)$ is the character trace on $H_2$.
The polynomial-space character at $\ell = 2$:
$\chi_{P_2}(\omega) = \sum_{a+b=2}\binom{a+2}{2}\binom{b+2}{2}\omega^{a-b}$
$= \binom{4}{2}\omega^2 + \binom{3}{2}\binom{3}{2}\omega^0 + \binom{2}{2}\binom{4}{2}\omega^{-2}$
$= 6\omega^2 + 9 + 6\omega = 9 + 6(\omega+\omega^2) = 9 - 6 = 3$.
Harmonic correction: $\chi_{H_2}(\omega) = \chi_{P_2}(\omega) - \chi_{P_0}(\omega) = 3 - 1 = 2$.
Therefore $d_2^{(0)} = (20 + 2 \cdot 2)/3 = 24/3 = 8$.
And $d_2^{\mathrm{ghost}} = 20 - 8 = 12$.  \qed
\end{proof}

\begin{remark}[Equidistribution at large $\ell$]
For $\ell \gg 1$: $d_\ell^{(0)} \to d_\ell/3$ (the $\mathbb{Z}_3$ characters equidistribute).
The ghost fraction $d_\ell^{\mathrm{ghost}}/d_\ell \to 2/3$.  The $\ell = 1$
exclusion ($d_1^{(0)} = 0$, ghost fraction $= 1$) is special to the first level.
This equidistribution is the mechanism behind the heavy-mode cancellation in the
cosmological constant derivation~\cite{leng2026}.
\end{remark}

%% ============================================================
\section{Summary}
%% ============================================================

On $S^5/\mathbb{Z}_3$ with the diagonal action:
\begin{enumerate}
\item All $d_1 = 6$ first-level harmonics are ghost modes (Theorem~\ref{thm:exclusion}).
\item The physical spectrum has a gap: no invariant modes at $\ell = 1$ (Corollary~\ref{cor:gap}).
\item No fundamental SU(3) triplet propagates: confinement (Corollary~\ref{cor:confinement}).
\item Chirality: $\eta = 2/9 \neq 0$ from the $\chi_1/\chi_2$ asymmetry (Corollary~\ref{cor:chirality}).
\end{enumerate}
These are representation-theoretic facts about $\mathbb{Z}_3 \hookrightarrow \mathrm{U}(3)$
acting on spherical harmonics.  No physical assumption is required.

%% ============================================================
\begin{thebibliography}{99}

\bibitem{ikeda1980}
A.~Ikeda, ``On the spectrum of the Laplacian on the spherical space forms,''
\textit{Osaka J.\ Math.}\ \textbf{17} (1980) 691--702.

\bibitem{leng2026eta}
J.~Leng, ``Eta invariants, Reidemeister torsion, and a ghost-mode identity
on the lens space $L(3;1,1,1)$,'' (2026).

\bibitem{leng2026}
J.~Leng, ``The Resolved Chord: The Theorem of Everything,'' v10 (2026).

\end{thebibliography}

\end{document}
